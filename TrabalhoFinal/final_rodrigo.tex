\documentclass[10pt,aspectratio=169]{beamer}

% =======================
% Pacotes básicos
% =======================
\usetheme{Madrid}
\usecolortheme{default}

\usepackage[utf8]{inputenc}
\usepackage[T1]{fontenc}
\usepackage[portuguese]{babel}
\usepackage{amsmath, amsfonts, amssymb}
\usepackage{graphicx}
\usepackage{booktabs}
\usepackage{multirow}

\title[Predição de Mudança de Estado]{Desenvolver um modelo gerador de séries temporais multivariadas
          para um compressor industrial offshore baseado em RNN e VAEs.}
\author{Rodrigo Petrus Domingues}
\institute{CPE727 -- Deep Learning}
\date{\today}

\begin{document}

% =======================
% Capa
% =======================
\begin{frame}
  \titlepage
\end{frame}

% =======================
% Sumário
% =======================
\begin{frame}{Sumário}
  \tableofcontents
\end{frame}

% =======================
% Seção: Motivação
% =======================
\section{Motivação e Objetivos}

\begin{frame}{Motivação e Objetivos do trabalho}
\noindent \textbf{Motivação:}
    \begin{itemize}
      \item Compressores industriais são ativos críticos (segurança, disponibilidade, custo).
      \item As séries temporais multivariadas de sensores refletem o estado operacional.
      \item Modelos generativos podem simular cenários operacionais variados.
      \item Gêmeos digitais e simulações realistas auxiliam na manutenção preditiva
    \end{itemize}

\noindent \textbf{Objetivos do trabalho:}

  \begin{itemize}
    \item Desenvolver \textbf{modelos geradores} de séries temporais multivariadas
          para um compressor industrial offshore.
    \item Aprender, de forma não supervisionada, a distribuição dos sinais de processo
          em janelas temporais.
    \item Gerar novas trajetórias plausíveis para:
      \begin{itemize}
        \item simulação de cenários,
        \item testes de algoritmos de monitoramento,
        \item análise de variabilidade operacional.
      \end{itemize}
     \end{itemize}
   \noindent \textbf{Comparar arquiteturas:}
        \begin{itemize}
          \item Modelos DEEP, LSTM, GRU, BiLSTM e BiGRU com GRU fuser, todos com otimização Adam;
          \item Variações de treinamento (warmup, scheduler, RAdam, AdamW, variational dropout, difusão, log-likelihood e layernorm).
        \end{itemize}


\end{frame}

% =======================
% Seção: Base de dados e pré-processamento
% =======================
\section{Base de dados e pré-processamento}


\begin{frame}{Base de dados e pré-processamento}

\begin{columns}[t]
  
  \begin{column}[t]{0.48\textwidth}
    \textbf{Base de dados}

    \vspace{0.2cm}
    Cognite — sinais reais de sensores de um compressor industrial offshore.

    \vspace{0.3cm}
    \centering
    \includegraphics[width=\linewidth]{/home/ferna/CPE727-2025-03/TrabalhoFinal/rodrigo/12series.png}

    \vspace{0.2cm}
    \scriptsize
    \textit{Exemplo de séries reais utilizadas no estudo (12 sinais simultâneos).}
  \end{column}

  \begin{column}[t]{0.48\textwidth}
    \textbf{Pré-processamento}

    \vspace{0.2cm}
    \begin{itemize}
      \item Estados operacionais:
        \begin{itemize}
          \item Estados originais: 6 (\(0\) a \(5\));
          \item Estados finais: 3 (normal, alerta, falha).
        \end{itemize}

      \item Normalização robusta (RobustScaler) por sensor;
      \item Série temporal com amostras a cada 5 minutos.
      \item Segmentação em janelas:
        \begin{itemize}
          \item \texttt{window\_size = 10}, \texttt{window\_step = 10}
          \item janelas não sobrepostas (modo gerador local).
        \end{itemize}
    \end{itemize}
  \end{column}

\end{columns}
\end{frame}


% =======================
% Seção: Formulação Teórica
% =======================
\section{Formulação do Problema}

\begin{frame}{Definição do problema (modo gerador)}

  \begin{itemize}
    \item Série multivariada: \(\mathbf{x}_t \in \mathbb{R}^{d}\), \(t = 1,\dots,T\), com \(d = 11\) sensores.
    \item Construímos janelas temporais:
      \[
        \mathbf{X} = (x_{t-L+1}, \dots, x_t) \in \mathbb{R}^{L \times d}
      \]
      com \(L = 10\) passos de tempo.
    \item Objetivo: aprender um \textbf{modelo gerador} para as janelas \(\mathbf{X}\):
      \[
        p_{\theta}(\mathbf{X}) \approx p_{\text{dados}}(\mathbf{X})
      \]
    \item Modelo latente (VAE + difusão):
      \begin{itemize}
        \item Encoder mapeia \(\mathbf{X}\) para um código latente \(z\);
        \item Decoder reconstrói \(\hat{\mathbf{X}}\) a partir de \(z\);
        \item O espaço latente é regularizado para permitir \textbf{amostrar novos} \(z\) e gerar janelas sintéticas.
      \end{itemize}
    \item Uso principal:
      \begin{itemize}
        \item Geração de séries realistas para simulação de cenários
              e testes de algoritmos de monitoramento.
      \end{itemize}
  \end{itemize}

  \vspace{0.3cm}
  \noindent O modelo aprende a \textbf{recriar e amostrar} janelas de sensores compatíveis
  com o comportamento real do compressor.
\end{frame}


\begin{frame}{Função de custo (modo gerador)}
\small

\begin{itemize}
  \item O modelo original admite uma função de custo multi-termo:
\[
\mathcal{L} =
\lambda_1 \mathcal{L}_{\text{rec}}
+ \lambda_2 \mathcal{L}_{\text{diff}}
+ \lambda_3 \mathcal{L}_{\text{time}}
+ \lambda_4 \mathcal{L}_{\text{mask}}
+ \boxed{\lambda_5 \mathcal{L}_{\text{VAE-}x}}
+ \lambda_6 \mathcal{L}_{\text{VAE-}\tau}
\]
\end{itemize}

\vspace{0.2cm}

\begin{itemize}
  \item \textbf{Neste experimento, o foco é exclusivamente gerador}:
    \[
      \lambda_5 = 1,
      \qquad
      \lambda_{1,2,3,4,6} = 0
    \]
  \item O treinamento é, portanto, \textbf{não supervisionado}, visando aprender:
    \[
      p_\theta(\mathbf{X}) \approx p_{\text{dados}}(\mathbf{X})
    \]
\end{itemize}

\vspace{0.2cm}

\small
\[
\mathcal{L}_{\text{VAE-}x}
=
\underbrace{
\mathbb{E}_{q(z|\mathbf{X})}
\big[
-\log p(\mathbf{X}\mid z)
\big]
}_{\text{erro de reconstrução}}
+
\beta\,
\underbrace{
\mathrm{KL}\big(q(z\mid\mathbf{X}) \,\|\, \mathcal{N}(0,I)\big)
}_{\text{regularização do espaço latente}}
\]

\vspace{0.2cm}

\begin{itemize}
  \item O termo de reconstrução garante fidelidade aos sinais reais.
  \item O termo KL força um espaço latente contínuo e amostrável.
  \item Isso permite \textbf{gerar novas janelas sintéticas} de sensores
        com propriedades estatísticas similares às observadas.
\end{itemize}

\end{frame}

\begin{frame}{Arquiteturas avaliadas}
  \begin{itemize}
    \item \textbf{TSDF\_DEEP}:
      \begin{itemize}
        \item Modelo feedforward sobre janelas (baseline não-recorrente).
      \end{itemize}
    \item \textbf{TSDF\_LSTM}:
      \begin{itemize}
        \item RNN LSTM unidirecional;
        \item Versão \textbf{BiLSTM} (\texttt{bi\_lstm=True}).
      \end{itemize}
    \item \textbf{TSDF\_GRU}:
      \begin{itemize}
        \item RNN GRU unidirecional;
        \item Versão \textbf{BiGRU} (\texttt{bi\_gru=True}).
      \end{itemize}
  \end{itemize}
  \vspace{0.2cm}
  \noindent \textbf{Hiperparâmetros principais}
    \begin{itemize}
      \item \texttt{in\_channels = 11}
      \item \texttt{hidden\_dim = 11 * 32}
      \item \texttt{status\_dim = 3}
      \item Janela: \texttt{window\_size = 10}, \texttt{window\_step = 10}
    \end{itemize}
  
\end{frame}


\begin{frame}{Arquitetura baseline: TSDF\_DEEP (feedforward)}
  \begin{itemize}
    \item Entrada: janela temporal \(\mathbf{X} = (x_1,\dots,x_L)\),
          com \(x_t \in \mathbb{R}^{11}\) e máscara \(m_t \in \{0,1\}^{11}\).
    \item Codificação inicial por MLP (por timestep):
      \[
        e_t = \phi\left( W_e [x_t \,\Vert\, m_t] + b_e \right),
        \quad t = 1,\dots,L
      \]
      onde \(\phi(\cdot)\) é uma ativação não linear (ReLU/GELU).
    \item Incorporação explícita do tempo:
      \[
        e'_t = \mathrm{MLP}\big([e_t \,\Vert\, t_t]\big)
      \]
      com \(t_t\) o timestamp normalizado do instante \(t\).
    \item Representação latente da janela:
      \begin{itemize}
        \item Obtida independentemente por timestep (sem estados recorrentes);
        \item Integra informação temporal apenas via timestamps explícitos.
      \end{itemize}
    \item Cabeças de saída (modo gerador):
      \begin{align*}
  \hat{x}_t &= f_{\theta}(e'_t) \quad \text{(reconstrução / geração)} 
\end{align*}
    \item Treinamento:
      \begin{itemize}
        \item Otimizador Adam;
        \item Mesmas funções de perda usadas nos modelos recorrentes.
      \end{itemize}
    
  \end{itemize}
\end{frame}



\begin{frame}{Arquitetura 1: LSTM}
  \begin{itemize}
    \item Entrada: janela temporal \(\mathbf{X} = (x_1,\dots,x_L)\), \(x_t \in \mathbb{R}^{11}\).
    \item Codificação temporal com LSTM unidirecional:
      \[
        (h_t, c_t) = \mathrm{LSTM}(x_t, h_{t-1}, c_{t-1}), \quad t = 1,\dots,L
      \]
   \item Representação latente da janela:
\[
  z = h_L \in \mathbb{R}^{H}
\]
\item $z$ representa um código latente contínuo da dinâmica temporal.
    \item Decoder:
\[
  \hat{\mathbf{X}} = p_\theta(\mathbf{X}\mid z)
\]
    \item Otimização:
      \begin{itemize}
        \item Otimizador Adam, minimizando perda conjunta (classificação de estado + regressão de tempo).
      \end{itemize}
  \end{itemize}
\end{frame}

\begin{frame}{Arquitetura 2: GRU}
  \begin{itemize}
    \item Mesma entrada: \(\mathbf{X} = (x_1,\dots,x_L)\), \(x_t \in \mathbb{R}^{11}\).
    \item Codificação temporal com GRU unidirecional:
      \[
        h_t = \mathrm{GRU}(x_t, h_{t-1}), \quad t = 1,\dots,L
      \]
    \item Representação latente da janela:
\[
  z = h_L \in \mathbb{R}^{H}
\]
\item $z$ representa um código latente contínuo da dinâmica temporal.
    \item Decoder:
\[
  \hat{\mathbf{X}} = p_\theta(\mathbf{X}\mid z)
\]
    \item Otimização:
      \begin{itemize}
        \item Otimizador Adam, com os mesmos hiperparâmetros do baseline Deep (learning rate, batch, etc.).
      \end{itemize}
  \end{itemize}
\end{frame}

\begin{frame}{Arquitetura 3: BiGRU + GRU fuser}
  \begin{itemize}
    \item Codificação da janela com BiGRU:
      \[
        h_t^{\rightarrow}, h_t^{\leftarrow} = \mathrm{BiGRU}(x_t, h_{t-1}^{\rightarrow}, h_{t+1}^{\leftarrow})
      \]
      \[
        z_k = [h_L^{\rightarrow} \,\Vert\, h_1^{\leftarrow}] \in \mathbb{R}^{2H}
      \]
      onde \(z_k\) é o embedding da \(k\)-ésima janela.
    \item Fusão temporal entre janelas com GRU fuser:
      \[
        m_k = \mathrm{GRU}_{\text{fuser}}(z_k, m_{k-1}), \quad k = 1,\dots,K
      \]
      \item Fuser captura dependências de longo alcance entre janelas,
      produzindo um latente $\mathbf{m}_K$ adequado à geração.
    \item Cabeças de saída a partir de \(m_K\):
     \[
\hat{\mathbf{X}} \sim p_\theta(\mathbf{X}\mid m_K)
\]
    \item Otimização:
      \begin{itemize}
        \item Adam sobre todos os parâmetros (BiGRU + GRU fuser + cabeças).
      \end{itemize}
  \end{itemize}
\end{frame}

\begin{frame}{Arquitetura 4: BiLSTM + GRU fuser}
  \begin{itemize}
    \item Codificação da janela com BiLSTM:
      \[
        (h_t^{\rightarrow}, c_t^{\rightarrow}), (h_t^{\leftarrow}, c_t^{\leftarrow})
          = \mathrm{BiLSTM}(x_t,\dots)
      \]
      \[
        z_k = [h_L^{\rightarrow} \,\Vert\, h_1^{\leftarrow}] \in \mathbb{R}^{2H}
      \]
    \item Fusão temporal entre janelas com GRU fuser:
      \[
        m_k = \mathrm{GRU}_{\text{fuser}}(z_k, m_{k-1}), \quad k = 1,\dots,K
      \]
        \item Fuser captura dependências de longo alcance entre janelas,
      produzindo um latente $\mathbf{m}_K$ adequado à geração.
    \item Cabeças de saída:
      \[
\hat{\mathbf{X}} \sim p_\theta(\mathbf{X}\mid m_K)
\]
    \item Otimização:
      \begin{itemize}
        \item Treino conjunto com Adam, mesma função de perda multi-tarefa.
      \end{itemize}
  \end{itemize}
\end{frame}


% =======================
% Seção: Arquiteturas
% =======================
\section{Arquiteturas e Variações}

\begin{frame}{Variações de treinamento - Otimização}
    

  \begin{itemize}
    \item \textbf{BiLSTM (GRU Fuser) + Warmup \& Scheduler}
      \begin{itemize}
        \item Aumenta gradualmente a taxa de aprendizado no início do treino, 
              evitando instabilidades e melhorando a convergência inicial.
        \item \texttt{warmup\_steps = 50}, \texttt{min\_lr\_factor = 0.01}
      \end{itemize}

    \item \textbf{BiLSTM (GRU Fuser)+ RAdam}
      \begin{itemize}
        \item Reduz a variância adaptativa do Adam nos primeiros passos, 
              tornando o treinamento mais estável.
        \item \texttt{optimizer\_name = 'radam'}
      \end{itemize}
 

     \end{itemize}
\end{frame}


\begin{frame}{Variações de treinamento — Regularização}
 \begin{itemize}

    \item \textbf{BiLSTM (GRU Fuser) + Variational Dropout}
      \begin{itemize}
        \item Regularização explícita no tempo,
              reduzindo overfitting sem quebrar dependências temporais.
        \item \texttt{variational\_dropout = 0.2}
      \end{itemize}

    \item \textbf{BiLSTM (GRU Fuser) + Difusão (missingness)}
      \begin{itemize}
        \item Regularização implícita via objetivos probabilísticos,
              forçando robustez a dados ausentes.
        \item \texttt{lam = [0., 0.0, 0.0, 0.05, 0.0, 0.95]}, \texttt{rebuild=True}
      \end{itemize}

       \item \textbf{BiLSTM (GRU Fuser) + LayerNorm}
      \begin{itemize}
        \item Estabiliza as ativações internas,
              com efeito regularizante indireto.
        \item \texttt{use\_layernorm = True}
      \end{itemize}

    \item \textbf{BiLSTM (GRU Fuser) + AdamW (Weight Decay)}
      \begin{itemize}
        \item Regularização explícita dos pesos,
              melhorando generalização.
        \item \texttt{optimizer\_name = 'adamw'}, \texttt{weight\_decay = 1e-4}
      \end{itemize}

  \end{itemize}
\end{frame}




% =======================
% Seção: Configuração Experimental
% =======================
\section{Configuração Experimental}

\begin{frame}{Configuração Experimental}
  \begin{itemize}
    \item Divisão treino/validação/teste: 60\% / 20\% / 20\%.
    \item Treinamento por 500 épocas, batch size 256.
    \item Paciênica de 50 épocas para early stopping.
    \item Função de perda multi-tarefa conforme descrito.
    \item Otimizador: Adam (exceto variações).
    \item Taxa de aprendizado inicial: \(1 \times 10^{-3}\).
    \item Early stopping baseado em MSE de validação.
    \item Avaliação final no conjunto de teste.
  \end{itemize}
\end{frame}
% =======================
% Seção: Resultados
% =======================
\section{Resultados}

\begin{frame}{Métricas utilizadas}
  \begin{itemize}
    \item \textbf{ELBO} (Evidence Lower Bound):
      \begin{itemize}
        \item Função objetivo otimizada no treinamento do modelo gerador;
    \item Maximiza a verossimilhança das séries observadas;
    \item Equilibra fidelidade das séries geradas (reconstrução)
          e regularização do espaço latente (KL).
      \end{itemize}

    \item \textbf{MSE} (Mean Squared Error):
      \begin{itemize}
         \item Erro médio de reconstrução das séries;
    \item Mede fidelidade gerativa.
      \end{itemize}

    \item \textbf{NLL} (Negative Log-Likelihood):
      \begin{itemize}
       \item Qualidade probabilística da geração;
    \item Penaliza variância mal calibrada.
      \end{itemize}

    \item \textbf{Cobertura 90\%} (\texttt{cov\_90}):
      \begin{itemize}
        \item Mede se os intervalos gerados contêm os valores reais;
    \item Avalia calibração do gerador.
      \end{itemize}

    \item \textbf{Largura do intervalo 90\%} (\texttt{width\_90}):
      \begin{itemize}
      \item Avalia sharpness do modelo;
    \item Intervalos estreitos com boa cobertura são desejáveis.
      \end{itemize}
  \end{itemize}
\end{frame}

\begin{frame}{Resultados quantitativos}
  \begin{table}[h]
    \centering
    \small
\begin{tabular}{lccccc}
\toprule
\textbf{Modelo} & \textbf{ELBO $\uparrow$} & \textbf{MSE $\downarrow$}
 & \textbf{NLL $\downarrow$} & \textbf{Cov90 $\uparrow$} & \textbf{Width90 $\downarrow$} \\
\midrule
DEEP & 1.217 & 0.693 ± 0.209 & 1.164 & 0.916 & 2.785 \\
BiLSTM (GRU Fuser) LayerNorm & 0.617 & 0.342 ± 0.140 & 0.514 & 0.907 & 2.281 \\
BiLSTM & 0.609 & 0.337 ± 0.133 & 0.527 & 0.901 & 2.269 \\
BiLSTM (GRU Fuser) AdamW & 0.601 & 0.341 ± 0.141 & 0.506 & 0.900 & 2.224 \\
BiLSTM (GRU Fuser) Warmup/Sched. & 0.598 & 0.354 ± 0.146 & 0.504 & 0.903 & 2.251 \\
BiLSTM (GRU Fuser) RAdam & 0.597 & 0.341 ± 0.139 & 0.511 & 0.898 & 2.234 \\
BiLSTM (GRU Fuser) VarDrop 0.2 & 0.595 & 0.317 ± 0.119 & 0.506 & 0.898 & 2.240 \\
BiGRU & 0.589 & 0.335 ± 0.136 & 0.499 & 0.909 & 2.267 \\
LSTM & 0.588 & 0.332 ± 0.129 & 0.513 & 0.900 & 2.251 \\
GRU & 0.558 & 0.300 ± 0.108 & 0.498 & 0.905 & 2.252 \\
BiLSTM (GRU Fuser) Difusão & 0.518 & 0.243 ± 0.067 & 0.462 & 0.902 & 2.200 \\
\bottomrule
\end{tabular}
    \caption{Comparação de desempenho entre modelos e variações.}
  \end{table}
  \scriptsize
Embora o baseline DEEP apresente ELBO elevado,
isso reflete um problema de reconstrução mais simples,
sem modelagem explícita da dependência temporal.
Modelos recorrentes exibem melhor qualidade gerativa,
com menor erro, melhor calibração e séries mais realistas.
\end{frame}



\begin{frame}{Curva de treinamento — Modelo campeão}

\begin{itemize}
  \item Evolução do ELBO ao longo das épocas para o modelo
        \textbf{BiLSTM (GRU Fuser + LayerNorm)}.
  \item Observa-se:
    \begin{itemize}
      \item Convergência mais rápida;
      \item Menor oscilação durante o treinamento;
      \item Melhor estabilidade em comparação aos modelos sem normalização.
    \end{itemize}
\end{itemize}

\begin{figure}
  \centering
  \includegraphics[width=0.70\linewidth]{/home/ferna/CPE727-2025-03/TrabalhoFinal/rodrigo/elbo.png}
  \caption{Evolução do ELBO durante o treinamento do modelo
  BiLSTM (GRU Fuser + LayerNorm).}
\end{figure}

\end{frame}

% =======================
% Seção: Discussão e Conclusões
% =======================
\section{Discussão e Conclusões}
\begin{frame}{Discussão dos resultados (modo gerador)}
  \begin{itemize}
    \item \textbf{Qualidade gerativa e ajuste probabilístico}:
      \begin{itemize}
        \item Modelos variacionais recorrentes superam a arquitetura feedforward (DEEP);
        \item \textbf{BiLSTM (GRU Fuser + LayerNorm)} apresenta o maior ELBO,
              indicando melhor equilíbrio entre reconstrução e regularização;
        \item Redução consistente de MSE e NLL em relação a LSTM/GRU unidirecionais.
      \end{itemize}

    \item \textbf{Trade-off sharpness vs. calibração}:
      \begin{itemize}
        \item Modelos com difusão e regularização variacional geram
              intervalos preditivos mais estreitos (menor width\_90);
        \item Cobertura próxima ao alvo de 90\% indica boa calibração probabilística;
        \item O modelo com difusão apresenta o menor MSE,
              ao custo de um ELBO inferior ao modelo campeão.
      \end{itemize}

    \item \textbf{Impacto arquitetural}:
      \begin{itemize}
        \item Bidirecionalidade melhora a qualidade das representações latentes;
        \item O \textit{GRU fuser} permite capturar dependências entre janelas consecutivas;
        \item A profundidade temporal é fundamental para geração realista de séries longas.
      \end{itemize}

    \item \textbf{Impacto do treinamento e regularização}:
      \begin{itemize}
        \item LayerNorm promove convergência mais estável e maior ELBO;
        \item Variational dropout e difusão atuam como regularizadores probabilísticos;
        \item AdamW e RAdam mostram ganhos marginais, dependentes do regime de treino.
      \end{itemize}
  \end{itemize}
\end{frame}

\begin{frame}{Conclusões (modo gerador)}
  \begin{itemize}
    \item O framework \textbf{TSDF} mostrou-se um modelo gerador eficaz
          para séries temporais multivariadas industriais,
          aprendendo distribuições realistas a partir de dados reais.

    \item Arquiteturas recorrentes superam o baseline feedforward:
      \begin{itemize}
        \item Modelos bidirecionais (BiLSTM / BiGRU) aprendem representações latentes
              mais ricas e consistentes;
        \item A fusão temporal entre janelas (GRU fuser) melhora a coerência
              de longo prazo das séries geradas.
      \end{itemize}

    \item A regularização probabilística é um fator chave:
      \begin{itemize}
        \item Difusão, variational dropout e LayerNorm
              aumentam estabilidade, calibração e qualidade gerativa;
        \item O modelo \textbf{BiLSTM (GRU Fuser + LayerNorm)}
              apresentou o melhor equilíbrio entre ajuste probabilístico e estabilidade.
      \end{itemize}

    \item O modelo gerador permite:
      \begin{itemize}
        \item Geração de séries sintéticas \textbf{realistas e calibradas};
        \item Simulação de regimes operacionais raros ou críticos;
        \item Suporte ao desenvolvimento e validação de algoritmos de monitoramento.
      \end{itemize}

    \item Este trabalho abre caminho para aplicações avançadas:
      \begin{itemize}
        \item Integração com gêmeos digitais industriais;
        \item Data augmentation para cenários de falha e degradação;
        \item Estudos de confiabilidade, robustez e prognóstico.
      \end{itemize}
  \end{itemize}
\end{frame}




\end{document}
