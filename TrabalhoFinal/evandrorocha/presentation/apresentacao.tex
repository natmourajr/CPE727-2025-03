\documentclass[aspectratio=169]{beamer}
\usetheme{Madrid}
\usecolortheme{default}

% Pacotes
\usepackage[utf8]{inputenc}
\usepackage[portuguese]{babel}
\usepackage{graphicx}
\usepackage{booktabs}
\usepackage{amsmath}
\usepackage{hyperref}

% Informações do título
\title{Detecção de Tuberculose em Radiografias de Tórax}
\subtitle{Utilizando Deep Learning e Redes Neurais Convolucionais}
\author{Seu Nome}
\institute{Universidade}
\date{\today}

\begin{document}

% Slide de título
\frame{\titlepage}

% Sumário
\begin{frame}{Sumário}
    \tableofcontents
\end{frame}

%===========================================
% 1. INTRODUÇÃO
%===========================================
\section{Introdução}

\begin{frame}{Contexto}
    \begin{columns}
        \column{0.5\textwidth}
        \textbf{Tuberculose: Um Problema Global}
        \begin{itemize}
            \item 10 milhões de casos/ano (OMS, 2022)
            \item 1,5 milhões de mortes/ano
            \item Principal causa de morte por doença infecciosa
            \item Diagnóstico precoce é crucial
        \end{itemize}
        
        \column{0.5\textwidth}
        \textbf{Desafios do Diagnóstico}
        \begin{itemize}
            \item Escassez de radiologistas
            \item Variabilidade inter-observador
            \item Tempo de análise
            \item Custo elevado
        \end{itemize}
    \end{columns}
\end{frame}

\begin{frame}{Motivação}
    \begin{block}{Por que Deep Learning?}
        \begin{itemize}
            \item Capacidade de aprender características complexas
            \item Alto desempenho em tarefas de visão computacional
            \item Potencial para auxiliar diagnóstico médico
            \item Redução de tempo e custo
        \end{itemize}
    \end{block}
    
    \vspace{0.5cm}
    
    \begin{alertblock}{Objetivo do Trabalho}
        Desenvolver e comparar modelos de Deep Learning para detecção automática de tuberculose em radiografias de tórax do dataset Shenzhen Hospital.
    \end{alertblock}
\end{frame}

%===========================================
% 2. PROBLEMA ABORDADO
%===========================================
\section{Problema Abordado}

\begin{frame}{Definição do Problema}
    \begin{block}{Problema de Classificação Binária}
        Dado uma radiografia de tórax (raio-X), classificar como:
        \begin{itemize}
            \item \textbf{Classe 0:} Normal (sem tuberculose)
            \item \textbf{Classe 1:} Tuberculose (presença da doença)
        \end{itemize}
    \end{block}
    
    \vspace{0.3cm}
    
    \begin{columns}
        \column{0.5\textwidth}
        \textbf{Desafios Técnicos:}
        \begin{itemize}
            \item Variabilidade nas imagens
            \item Sutileza dos padrões
            \item Dataset limitado
            \item Desbalanceamento de classes
        \end{itemize}
        
        \column{0.5\textwidth}
        \textbf{Requisitos:}
        \begin{itemize}
            \item Alta sensibilidade (recall)
            \item Alta especificidade
            \item Interpretabilidade
            \item Eficiência computacional
        \end{itemize}
    \end{columns}
\end{frame}

\begin{frame}{Dataset Shenzhen Hospital}
    \begin{columns}
        \column{0.6\textwidth}
        \textbf{Características do Dataset:}
        \begin{itemize}
            \item \textbf{Total:} 662 radiografias
            \item \textbf{Normal:} 326 imagens
            \item \textbf{Tuberculose:} 336 imagens
            \item \textbf{Formato:} PNG, grayscale
            \item \textbf{Resolução:} Variável
            \item \textbf{Fonte:} NIH/NLM
        \end{itemize}
        
        \column{0.4\textwidth}
        \textbf{Divisão dos Dados:}
        \begin{table}
            \small
            \begin{tabular}{lc}
                \toprule
                \textbf{Conjunto} & \textbf{\%} \\
                \midrule
                Treino & 70\% \\
                Validação & 15\% \\
                Teste & 15\% \\
                \bottomrule
            \end{tabular}
        \end{table}
        
        \vspace{0.3cm}
        \textbf{Balanceamento:} 1.03 (quase perfeito)
    \end{columns}
\end{frame}

%===========================================
% 3. REVISÃO BIBLIOGRÁFICA
%===========================================
\section{Revisão Bibliográfica}

\begin{frame}{Estado da Arte}
    \begin{block}{Trabalhos Relacionados}
        \begin{itemize}
            \item \textbf{Lakhani \& Sundaram (2017):} ResNet-50 com AUC 0.99
            \item \textbf{Hwang et al. (2016):} CNN customizada, acurácia 95.9\%
            \item \textbf{Stirenko et al. (2018):} Ensemble de CNNs, F1-score 0.93
            \item \textbf{Rajaraman et al. (2018):} Transfer learning, sensibilidade 97\%
        \end{itemize}
    \end{block}
    
    \vspace{0.3cm}
    
    \begin{exampleblock}{Tendências Observadas}
        \begin{itemize}
            \item Uso de transfer learning com ImageNet
            \item Arquiteturas profundas (ResNet, DenseNet)
            \item Data augmentation para aumentar dataset
            \item Ensemble methods para melhor performance
        \end{itemize}
    \end{exampleblock}
\end{frame}

\begin{frame}{Base Teórica}
    \begin{columns}
        \column{0.5\textwidth}
        \textbf{Redes Neurais Convolucionais (CNNs)}
        \begin{itemize}
            \item Camadas convolucionais
            \item Pooling layers
            \item Fully connected layers
            \item Ativação ReLU
            \item Dropout para regularização
        \end{itemize}
        
        \vspace{0.3cm}
        
        \textbf{Transfer Learning}
        \begin{itemize}
            \item Pesos pré-treinados (ImageNet)
            \item Fine-tuning
            \item Feature extraction
        \end{itemize}
        
        \column{0.5\textwidth}
        \textbf{Arquiteturas Utilizadas}
        \begin{itemize}
            \item \textbf{ResNet-50:} Skip connections
            \item \textbf{DenseNet-121:} Dense connections
            \item \textbf{EfficientNet-B0:} Compound scaling
            \item \textbf{SimpleCNN\_TB:} Arquitetura customizada
        \end{itemize}
        
        \vspace{0.3cm}
        
        \textbf{MLP com Features}
        \begin{itemize}
            \item Deep learning features
            \item Handcrafted features
        \end{itemize}
    \end{columns}
\end{frame}

%===========================================
% 4. MÉTODO PROPOSTO
%===========================================
\section{Método Proposto}

\begin{frame}{Pipeline de Processamento}
    \begin{enumerate}
        \item \textbf{Pré-processamento}
        \begin{itemize}
            \item Redimensionamento: 224×224 pixels
            \item Normalização ImageNet: $\mu=[0.485, 0.456, 0.406]$, $\sigma=[0.229, 0.224, 0.225]$
            \item Conversão para RGB (3 canais)
        \end{itemize}
        
        \item \textbf{Data Augmentation (apenas treino)}
        \begin{itemize}
            \item Horizontal flip (p=0.5)
            \item Random brightness/contrast (p=0.3)
            \item Rotation/scale/shift (p=0.5, $\pm15°$)
        \end{itemize}
        
        \item \textbf{Treinamento}
        \begin{itemize}
            \item Otimizador: Adam ($lr=10^{-4}$, $wd=10^{-5}$)
            \item Loss: Cross-Entropy
            \item Batch size: 16-32 (dependendo da GPU)
            \item Early stopping: patience=10
        \end{itemize}
    \end{enumerate}
\end{frame}

\begin{frame}{Modelos Implementados}
    \begin{table}
        \small
        \begin{tabular}{lccl}
            \toprule
            \textbf{Modelo} & \textbf{Parâmetros} & \textbf{Pretrained} & \textbf{Características} \\
            \midrule
            ResNet-50 & 25.6M & ImageNet & Skip connections, 50 camadas \\
            DenseNet-121 & 8.0M & ImageNet & Dense connections \\
            EfficientNet-B0 & 5.3M & ImageNet & Compound scaling \\
            SimpleCNN\_TB & 2.1M & Não & Arquitetura customizada \\
            MLP (Deep) & 0.5M & Não & Features ResNet-50 \\
            MLP (Manual) & 0.1M & Não & 512 features manuais \\
            \bottomrule
        \end{tabular}
    \end{table}
    
    \vspace{0.3cm}
    
    \begin{block}{Estratégia de Avaliação}
        \begin{itemize}
            \item Métricas: Acurácia, Precisão, Recall, F1-Score, AUC-ROC
            \item Validação cruzada estratificada
            \item Análise de matriz de confusão
        \end{itemize}
    \end{block}
\end{frame}

\begin{frame}{Ambiente de Desenvolvimento}
    \begin{columns}
        \column{0.5\textwidth}
        \textbf{Hardware:}
        \begin{itemize}
            \item GPU: NVIDIA RTX 5060 Ti (16GB)
            \item CUDA: 12.6
            \item RAM: 32GB
        \end{itemize}
        
        \vspace{0.3cm}
        
        \textbf{Software:}
        \begin{itemize}
            \item Python 3.11
            \item PyTorch 2.9.1
            \item Docker + Docker Compose
            \item Jupyter Lab
        \end{itemize}
        
        \column{0.5\textwidth}
        \textbf{Bibliotecas Principais:}
        \begin{itemize}
            \item \texttt{torch} - Deep learning
            \item \texttt{torchvision} - Modelos pré-treinados
            \item \texttt{albumentations} - Augmentation
            \item \texttt{scikit-learn} - Métricas
            \item \texttt{matplotlib} - Visualização
        \end{itemize}
        
        \vspace{0.3cm}
        
        \textbf{Tempo de Treinamento:}
        \begin{itemize}
            \item ResNet-50: $\sim$1-2 min/época
            \item 50 épocas: $\sim$1.5 horas
        \end{itemize}
    \end{columns}
\end{frame}

%===========================================
% 5. RESULTADOS
%===========================================
\section{Resultados Obtidos}

\begin{frame}{Resultados Comparativos}
    \begin{table}
        \footnotesize
        \begin{tabular}{lccccc}
            \toprule
            \textbf{Modelo} & \textbf{Acurácia} & \textbf{Precisão} & \textbf{Recall} & \textbf{F1-Score} & \textbf{AUC-ROC} \\
            \midrule
            ResNet-50 & 0.XXX & 0.XXX & 0.XXX & 0.XXX & 0.XXX \\
            DenseNet-121 & 0.XXX & 0.XXX & 0.XXX & 0.XXX & 0.XXX \\
            EfficientNet-B0 & 0.XXX & 0.XXX & 0.XXX & 0.XXX & 0.XXX \\
            SimpleCNN\_TB & 0.XXX & 0.XXX & 0.XXX & 0.XXX & 0.XXX \\
            MLP (Deep) & 0.XXX & 0.XXX & 0.XXX & 0.XXX & 0.XXX \\
            MLP (Manual) & 0.XXX & 0.XXX & 0.XXX & 0.XXX & 0.XXX \\
            \bottomrule
        \end{tabular}
    \end{table}
    
    \vspace{0.3cm}
    
    \begin{alertblock}{Nota}
        Preencher com os resultados reais após conclusão do treinamento.
        Atualmente em execução: \texttt{train\_all\_windows.bat}
    \end{alertblock}
\end{frame}

\begin{frame}{Análise dos Resultados}
    \begin{columns}
        \column{0.5\textwidth}
        \textbf{Observações Esperadas:}
        \begin{itemize}
            \item Transfer learning deve superar modelos from scratch
            \item ResNet-50/DenseNet: melhor performance
            \item SimpleCNN\_TB: boa relação custo-benefício
            \item MLP: baseline para comparação
        \end{itemize}
        
        \column{0.5\textwidth}
        \textbf{Métricas Importantes:}
        \begin{itemize}
            \item \textbf{Recall:} Minimizar falsos negativos
            \item \textbf{Precisão:} Minimizar falsos positivos
            \item \textbf{F1-Score:} Balanço geral
            \item \textbf{AUC-ROC:} Capacidade discriminativa
        \end{itemize}
    \end{columns}
    
    \vspace{0.5cm}
    
    \begin{exampleblock}{Matriz de Confusão}
        Análise detalhada de verdadeiros positivos, verdadeiros negativos, falsos positivos e falsos negativos para cada modelo.
    \end{exampleblock}
\end{frame}

\begin{frame}{Curvas ROC}
    \begin{center}
        % Placeholder para gráfico
        \textbf{[Inserir gráfico de curvas ROC comparativas]}
        
        \vspace{1cm}
        
        Comparação visual da capacidade discriminativa de cada modelo através das curvas ROC (Receiver Operating Characteristic).
        
        \vspace{0.5cm}
        
        Quanto maior a área sob a curva (AUC), melhor o desempenho do modelo.
    \end{center}
\end{frame}

%===========================================
% 6. CONCLUSÕES
%===========================================
\section{Conclusões}

\begin{frame}{Conclusões}
    \begin{block}{Principais Contribuições}
        \begin{itemize}
            \item Implementação e comparação de 6 modelos diferentes
            \item Pipeline completo de processamento e treinamento
            \item Ambiente reproduzível com Docker
            \item Documentação detalhada do projeto
        \end{itemize}
    \end{block}
    
    \vspace{0.3cm}
    
    \begin{block}{Lições Aprendidas}
        \begin{itemize}
            \item Importância do pré-processamento adequado
            \item Transfer learning acelera convergência
            \item Data augmentation crucial para datasets pequenos
            \item Balanceamento de classes impacta métricas
        \end{itemize}
    \end{block}
\end{frame}

\begin{frame}{Trabalhos Futuros}
    \begin{enumerate}
        \item \textbf{Expansão do Dataset}
        \begin{itemize}
            \item Incorporar outros datasets públicos
            \item Aumentar diversidade de casos
        \end{itemize}
        
        \item \textbf{Técnicas Avançadas}
        \begin{itemize}
            \item Attention mechanisms
            \item Grad-CAM para interpretabilidade
            \item Ensemble methods
        \end{itemize}
        
        \item \textbf{Aplicação Prática}
        \begin{itemize}
            \item Interface web para uso clínico
            \item Integração com PACS
            \item Validação com radiologistas
        \end{itemize}
        
        \item \textbf{Otimização}
        \begin{itemize}
            \item Quantização de modelos
            \item Deployment em edge devices
        \end{itemize}
    \end{enumerate}
\end{frame}

\begin{frame}{Referências}
    \footnotesize
    \begin{thebibliography}{99}
        \bibitem{lakhani2017}
        Lakhani, P., \& Sundaram, B. (2017). Deep learning at chest radiography: automated classification of pulmonary tuberculosis by using convolutional neural networks. \textit{Radiology}, 284(2), 574-582.
        
        \bibitem{hwang2016}
        Hwang, S., et al. (2016). A novel approach for tuberculosis screening based on deep convolutional neural networks. \textit{Medical Imaging 2016}.
        
        \bibitem{stirenko2018}
        Stirenko, S., et al. (2018). Chest X-ray analysis of tuberculosis by deep learning with segmentation and augmentation. \textit{IEEE IDAACS}, 422-428.
        
        \bibitem{rajaraman2018}
        Rajaraman, S., et al. (2018). Pre-trained convolutional neural networks as feature extractors toward improved malaria parasite detection. \textit{PeerJ}, 6, e4568.
    \end{thebibliography}
\end{frame}

\begin{frame}[standout]
    \Huge Obrigado!
    
    \vspace{1cm}
    
    \Large Perguntas?
    
    \vspace{1cm}
    
    \normalsize
    \texttt{GitHub:} \url{https://github.com/seu-usuario/tuberculosis-detection}
    
    \texttt{Email:} seu.email@universidade.edu.br
\end{frame}

\end{document}
