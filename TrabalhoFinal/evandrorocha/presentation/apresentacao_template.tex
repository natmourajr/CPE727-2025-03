\documentclass{beamer}

\usetheme{default}
\usepackage[utf8]{inputenc}
\usepackage{tikz}
\usepackage{tabularx}
\usetikzlibrary{positioning}
\usepackage{amsmath}
\usepackage{booktabs}
\usepackage{multirow}
\usepackage{graphicx}
\usepackage{hyperref}

\setbeamertemplate{footline}[frame number]
\setbeamertemplate{frametitle}[default][center]

\mode<presentation>{
\usetheme{Dresden}
\usecolortheme{seagull}
}

\setbeamertemplate{headline}{}

\definecolor{peeblue}{RGB}{1,123,165}
\setbeamercolor*{palette primary}{fg=black,bg=peeblue!80}
\setbeamercolor*{palette secondary}{fg=black,bg=peeblue!80!gray!80}
\setbeamercolor*{palette tertiary}{fg=black,bg=peeblue!100}
\setbeamercolor*{palette quaternary}{fg=black,bg=peeblue!110}

\usebackgroundtemplate
{%
    \begin{picture}(210,40)(-5,2)
    \includegraphics[width=0.07\paperwidth,keepaspectratio]{imgs/pee-logo-short.png}
    \end{picture}%
}

 \AtBeginSection[]
  {
    \begin{frame}
      \centering
      \huge\insertsectionhead
    \end{frame}
  }

\title{Detecção de Tuberculose em Radiografias de Tórax}
\subtitle{Utilizando Deep Learning e Transfer Learning}
\author{Evandro Rocha}
\date{\today}

\institute
{
  Universidade Federal do Rio de Janeiro\\
  UFRJ/COPPE/PEE
}

\begin{document}

{
\usebackgroundtemplate{
    \begin{picture}(210,55)(-5,0)
    \includegraphics[height=0.14\paperwidth,keepaspectratio]{imgs/pee-logo.png}
    \end{picture}%
    \begin{picture}(210,55)(-28,2)
    \includegraphics[height=0.14\paperwidth,keepaspectratio]{imgs/coppe-logo.pdf}
    \end{picture}
}
\begin{frame}
  \bigskip\bigskip\bigskip\bigskip
  \titlepage
\end{frame}
}

\begin{frame}{Agenda}
  \tableofcontents
\end{frame}

%===========================================
% 1. INTRODUÇÃO
%===========================================
\section{Introdução}

\begin{frame}{Contexto}
    \begin{columns}
        \column{0.5\textwidth}
        \textbf{Tuberculose: Um Problema Global}
        \begin{itemize}
            \item 10 milhões de casos/ano (OMS, 2022)
            \item 1,5 milhões de mortes/ano
            \item Principal causa de morte por doença infecciosa
            \item Diagnóstico precoce é crucial
        \end{itemize}
        
        \column{0.5\textwidth}
        \textbf{Desafios do Diagnóstico}
        \begin{itemize}
            \item Escassez de radiologistas
            \item Variabilidade inter-observador
            \item Tempo de análise
            \item Custo elevado
        \end{itemize}
    \end{columns}
\end{frame}

\begin{frame}{Motivação}
    \begin{block}{Por que Deep Learning?}
        \begin{itemize}
            \item Capacidade de aprender características complexas
            \item Alto desempenho em tarefas de visão computacional
            \item Potencial para auxiliar diagnóstico médico
            \item Redução de tempo e custo
        \end{itemize}
    \end{block}
    
    \vspace{0.5cm}
    
    \begin{alertblock}{Objetivo do Trabalho}
        Desenvolver e avaliar modelos de Deep Learning para detecção automática de tuberculose em radiografias de tórax do dataset Shenzhen Hospital.
    \end{alertblock}
\end{frame}

%===========================================
% 2. PROBLEMA ABORDADO
%===========================================
\section{Problema Abordado}

\begin{frame}{Definição do Problema}
    \begin{block}{Problema de Classificação Binária}
        Dado uma radiografia de tórax (raio-X), classificar como:
        \begin{itemize}
            \item \textbf{Classe 0:} Normal (sem tuberculose)
            \item \textbf{Classe 1:} Tuberculose (presença da doença)
        \end{itemize}
    \end{block}
    
    \vspace{0.3cm}
    
    \begin{columns}
        \column{0.5\textwidth}
        \textbf{Desafios Técnicos:}
        \begin{itemize}
            \item Variabilidade nas imagens
            \item Sutileza dos padrões
            \item Dataset limitado (662 imagens)
            \item Desbalanceamento de classes
        \end{itemize}
        
        \column{0.5\textwidth}
        \textbf{Requisitos:}
        \begin{itemize}
            \item Alta sensibilidade (recall)
            \item Alta especificidade
            \item Interpretabilidade
            \item Eficiência computacional
        \end{itemize}
    \end{columns}
\end{frame}

\begin{frame}{Dataset Shenzhen Hospital}
    \begin{columns}
        \column{0.6\textwidth}
        \textbf{Características do Dataset:}
        \begin{itemize}
            \item \textbf{Total:} 662 radiografias
            \item \textbf{Normal:} 326 imagens
            \item \textbf{Tuberculose:} 336 imagens
            \item \textbf{Formato:} PNG, grayscale
            \item \textbf{Resolução:} Variável
            \item \textbf{Fonte:} NIH/NLM
        \end{itemize}
        
        \column{0.4\textwidth}
        \textbf{Divisão dos Dados:}
        \begin{table}
            \small
            \begin{tabular}{lc}
                \toprule
                \textbf{Conjunto} & \textbf{\%} \\
                \midrule
                Treino & 70\% \\
                Validação & 15\% \\
                Teste & 15\% \\
                \bottomrule
            \end{tabular}
        \end{table}
        
        \vspace{0.3cm}
        \textbf{Balanceamento:} 1.03 (quase perfeito)
    \end{columns}
\end{frame}

%===========================================
% 3. MÉTODO PROPOSTO
%===========================================
\section{Método Proposto}

\begin{frame}{Pipeline de Processamento}
    \begin{enumerate}
        \item \textbf{Pré-processamento}
        \begin{itemize}
            \item Redimensionamento: 224×224 pixels
            \item Normalização ImageNet: $\mu=[0.485, 0.456, 0.406]$, $\sigma=[0.229, 0.224, 0.225]$
            \item Conversão para RGB (3 canais)
        \end{itemize}
        
        \item \textbf{Data Augmentation (apenas treino)}
        \begin{itemize}
            \item Horizontal flip (p=0.5)
            \item Random brightness/contrast (p=0.3)
            \item Rotation/scale/shift (p=0.5, $\pm15°$)
        \end{itemize}
        
        \item \textbf{Transfer Learning}
        \begin{itemize}
            \item Backbone congelado (feature extraction)
            \item Apenas classificador treinado
            \item Reduz overfitting em datasets pequenos
        \end{itemize}
    \end{enumerate}
\end{frame}

\begin{frame}{Transfer Learning: Feature Extraction}
    \begin{block}{Estratégia Adotada}
        \begin{itemize}
            \item \textbf{Backbone:} Pesos pré-treinados no ImageNet (congelados)
            \item \textbf{Classificador:} Treinado do zero para TB vs Normal
            \item \textbf{Parâmetros treináveis:} $\sim$1M (vs 25M total)
        \end{itemize}
    \end{block}
    
    \vspace{0.3cm}
    
    \begin{columns}
        \column{0.5\textwidth}
        \textbf{Vantagens:}
        \begin{itemize}
            \item Evita overfitting
            \item Treina mais rápido
            \item Usa menos memória GPU
            \item Aproveita features do ImageNet
        \end{itemize}
        
        \column{0.5\textwidth}
        \textbf{Treinamento:}
        \begin{itemize}
            \item Otimizador: Adam
            \item Learning rate: $10^{-4}$
            \item Batch size: 16
            \item Early stopping: patience=10
        \end{itemize}
    \end{columns}
\end{frame}

\begin{frame}{Modelos Avaliados}
    \begin{table}
        \small
        \begin{tabular}{lccc}
            \toprule
            \textbf{Modelo} & \textbf{Parâmetros} & \textbf{Pretrained} & \textbf{Características} \\
            \midrule
            ResNet-50 & 25.6M & ImageNet & Skip connections, 50 camadas \\
            DenseNet-121 & 8.0M & ImageNet & Dense connections \\
            EfficientNet-B0 & 5.3M & ImageNet & Compound scaling \\
            \bottomrule
        \end{tabular}
    \end{table}
    
    \vspace{0.5cm}
    
    \begin{block}{Métricas de Avaliação}
        \begin{itemize}
            \item \textbf{Sensibilidade:} Detectar casos de TB
            \item \textbf{Especificidade:} Identificar casos normais
            \item \textbf{AUC-ROC:} Capacidade discriminativa
            \item \textbf{F1-Score:} Balanço geral
        \end{itemize}
    \end{block}
\end{frame}

\begin{frame}{SimpleCNN: Baseline Tradicional}
    \begin{columns}
        \column{0.5\textwidth}
        \textbf{Arquitetura:}
        \begin{itemize}
            \item CNN treinada do zero
            \item 4 blocos convolucionais
            \item 1.2M parâmetros
            \item Sem pré-treinamento
        \end{itemize}
        
        \vspace{0.3cm}
        
        \textbf{Objetivo:}
        \begin{itemize}
            \item Servir como baseline
            \item Demonstrar valor do transfer learning
            \item Comparação justa
        \end{itemize}
        
        \column{0.5\textwidth}
        \textbf{Estrutura:}
        \begin{enumerate}
            \item Conv2D (32 filtros)
            \item Conv2D (64 filtros)
            \item Conv2D (128 filtros)
            \item Conv2D (256 filtros)
            \item Global Average Pooling
            \item Classificador (2 classes)
        \end{enumerate}
        
        \vspace{0.3cm}
        
        \begin{alertblock}{Limitações}
            Dataset pequeno (662 imagens) dificulta treinamento do zero
        \end{alertblock}
    \end{columns}
\end{frame}

\begin{frame}{Por que SimpleCNN como Baseline?}
    \begin{block}{Justificativa Acadêmica}
        \begin{itemize}
            \item \textbf{Comparação justa}: Mesma tarefa, mesmos dados
            \item \textbf{Demonstra valor}: Transfer learning vs treinar do zero
            \item \textbf{Baseline tradicional}: CNN sem conhecimento prévio
            \item \textbf{Evidência empírica}: Quantifica ganho do ImageNet
        \end{itemize}
    \end{block}
    
    \vspace{0.3cm}
    
    \begin{columns}
        \column{0.5\textwidth}
        \textbf{SimpleCNN (do zero):}
        \begin{itemize}
            \item Aprende tudo do zero
            \item Precisa de mais épocas (100)
            \item Risco de overfitting
            \item Performance esperada: 75-80\%
        \end{itemize}
        
        \column{0.5\textwidth}
        \textbf{Pré-treinadas (ImageNet):}
        \begin{itemize}
            \item Já conhecem features básicas
            \item Menos épocas (50)
            \item Menos overfitting
            \item Performance: 86-96\%
        \end{itemize}
    \end{columns}
\end{frame}

%===========================================
% 4. RESULTADOS
%===========================================
\section{Resultados Obtidos}

\begin{frame}{Resultados - ResNet50}
    \begin{table}
        \begin{tabular}{lc}
            \toprule
            \textbf{Métrica} & \textbf{Valor} \\
            \midrule
            Acurácia & 91.00\% \\
            Precisão & 91.84\% \\
            Recall (Sensibilidade) & 90.00\% \\
            F1-Score & 90.91\% \\
            AUC-ROC & \textbf{96.04\%} \\
            \bottomrule
        \end{tabular}
    \end{table}
    
    \vspace{0.5cm}
    
    \begin{exampleblock}{Destaque}
        AUC-ROC de 96.04\% indica excelente capacidade discriminativa entre casos de TB e normais.
    \end{exampleblock}
\end{frame}

\begin{frame}{Resultados - DenseNet121}
    \begin{table}
        \begin{tabular}{lc}
            \toprule
            \textbf{Métrica} & \textbf{Valor} \\
            \midrule
            Acurácia & 85.00\% \\
            Precisão & 87.23\% \\
            Recall (Sensibilidade) & 82.00\% \\
            F1-Score & 84.54\% \\
            AUC-ROC & 86.32\% \\
            \bottomrule
        \end{tabular}
    \end{table}
    
    \vspace{0.3cm}
    
    \begin{columns}
        \column{0.5\textwidth}
        \textbf{Matriz de Confusão:}
        \begin{table}
            \footnotesize
            \begin{tabular}{cc|cc}
                \multicolumn{2}{c}{} & \multicolumn{2}{c}{\textbf{Predito}} \\
                \multicolumn{2}{c|}{} & Normal & TB \\
                \cline{2-4}
                \multirow{2}{*}{\rotatebox{90}{\textbf{Real}}} & Normal & 44 & 6 \\
                & TB & 9 & 41 \\
            \end{tabular}
        \end{table}
        
        \column{0.5\textwidth}
        \textbf{Métricas Clínicas:}
        \begin{itemize}
            \item Sensibilidade: 82\%
            \item Especificidade: 88\%
            \item Convergiu em 13 épocas
            \item Modelo mais leve (8M parâmetros)
        \end{itemize}
    \end{columns}
\end{frame}

\begin{frame}{Comparação: ResNet50 vs DenseNet121}
    \begin{table}
        \small
        \begin{tabular}{lccc}
            \toprule
            \textbf{Métrica} & \textbf{ResNet50} & \textbf{DenseNet121} & \textbf{Diferença} \\
            \midrule
            Acurácia & \textbf{91.00\%} & 85.00\% & +6.0\% \\
            Precisão & \textbf{91.84\%} & 87.23\% & +4.6\% \\
            Sensibilidade & \textbf{90.00\%} & 82.00\% & +8.0\% \\
            Especificidade & \textbf{92.00\%} & 88.00\% & +4.0\% \\
            F1-Score & \textbf{90.91\%} & 84.54\% & +6.4\% \\
            AUC-ROC & \textbf{96.04\%} & 86.32\% & +9.7\% \\
            \midrule
            Parâmetros & 25.6M & \textbf{8.0M} & -69\% \\
            Épocas & 25 & \textbf{13} & -48\% \\
            \bottomrule
        \end{tabular}
    \end{table}
    
    \vspace{0.3cm}
    
    \begin{exampleblock}{Conclusão}
        \textbf{ResNet50 apresentou desempenho superior} em todas as métricas clínicas:
        \begin{itemize}
            \item +9.7\% em AUC-ROC (capacidade discriminativa)
            \item +8.0\% em Sensibilidade (detecção de TB)
            \item +4.0\% em Especificidade (identificação de normais)
        \end{itemize}
        DenseNet121 é mais eficiente (menos parâmetros, convergência mais rápida), mas ResNet50 oferece melhor performance para aplicação clínica.
    \end{exampleblock}
\end{frame}

\begin{frame}{Comparação Completa: Três CNNs Pré-treinadas}
    \begin{table}
        \footnotesize
        \begin{tabular}{lcccc}
            \toprule
            \textbf{Métrica} & \textbf{ResNet50} & \textbf{DenseNet121} & \textbf{EfficientNet-B0} & \textbf{Melhor} \\
            \midrule
            Sensibilidade & \textbf{90.00\%} & 82.00\% & 80.00\% & ResNet50 \\
            Especificidade & 92.00\% & 88.00\% & \textbf{94.00\%} & EfficientNet \\
            AUC-ROC & \textbf{96.04\%} & 86.32\% & 89.48\% & ResNet50 \\
            Acurácia & \textbf{91.00\%} & 85.00\% & 87.00\% & ResNet50 \\
            Precisão & 91.84\% & 87.23\% & \textbf{93.02\%} & EfficientNet \\
            F1-Score & \textbf{90.91\%} & 84.54\% & 86.02\% & ResNet50 \\
            \midrule
            Parâmetros & 25.6M & \textbf{8.0M} & \textbf{5.3M} & EfficientNet \\
            Épocas & 25 & \textbf{13} & 16 & DenseNet \\
            FN (Falsos Neg.) & \textbf{5} & 9 & 10 & ResNet50 \\
            FP (Falsos Pos.) & 4 & 6 & \textbf{3} & EfficientNet \\
            \bottomrule
        \end{tabular}
    \end{table}
    
    \vspace{0.3cm}
    
    \begin{columns}
        \column{0.5\textwidth}
        \textbf{Análise por Modelo:}
        \begin{itemize}
            \item \textbf{ResNet50}: Melhor performance geral
            \item \textbf{EfficientNet}: Melhor especificidade
            \item \textbf{DenseNet}: Mais eficiente
        \end{itemize}
        
        \column{0.5\textwidth}
        \textbf{Trade-offs:}
        \begin{itemize}
            \item Performance vs Eficiência
            \item Sensibilidade vs Especificidade
            \item Convergência vs Acurácia
        \end{itemize}
    \end{columns}
\end{frame}

\begin{frame}{Análise Detalhada por Modelo}
    \begin{columns}
        \column{0.33\textwidth}
        \textbf{ResNet50}
        \begin{block}{Pontos Fortes}
            \begin{itemize}
                \item ✓ Melhor AUC-ROC (96\%)
                \item ✓ Melhor sensibilidade (90\%)
                \item ✓ Balanceamento ideal
                \item ✓ Menos FN (5)
            \end{itemize}
        \end{block}
        \begin{alertblock}{Limitações}
            \begin{itemize}
                \item Mais parâmetros (25.6M)
                \item Mais épocas (25)
            \end{itemize}
        \end{alertblock}
        
        \column{0.33\textwidth}
        \textbf{DenseNet121}
        \begin{block}{Pontos Fortes}
            \begin{itemize}
                \item ✓ Eficiente (8M)
                \item ✓ Rápida convergência (13)
                \item ✓ Boa performance geral
            \end{itemize}
        \end{block}
        \begin{alertblock}{Limitações}
            \begin{itemize}
                \item Menor AUC-ROC (86\%)
                \item Mais FN (9)
                \item Sensibilidade baixa (82\%)
            \end{itemize}
        \end{alertblock}
        
        \column{0.33\textwidth}
        \textbf{EfficientNet-B0}
        \begin{block}{Pontos Fortes}
            \begin{itemize}
                \item ✓ Mais leve (5.3M)
                \item ✓ Melhor especificidade (94\%)
                \item ✓ Menos FP (3)
                \item ✓ Alta precisão (93\%)
            \end{itemize}
        \end{block}
        \begin{alertblock}{Limitações}
            \begin{itemize}
                \item Sensibilidade baixa (80\%)
                \item Mais FN (10)
            \end{itemize}
        \end{alertblock}
    \end{columns}
\end{frame}

\begin{frame}{Sensibilidade e Especificidade - ResNet50}
    \begin{columns}
        \column{0.5\textwidth}
        \textbf{Métricas Clínicas:}
        \begin{itemize}
            \item \textbf{Sensibilidade:} 90.00\%
            \begin{itemize}
                \item Detecta 9 em cada 10 casos de TB
                \item TP/(TP+FN) = 45/(45+5)
            \end{itemize}
            \item \textbf{Especificidade:} 92.00\%
            \begin{itemize}
                \item Identifica 92\% dos normais
                \item TN/(TN+FP) = 46/(46+4)
            \end{itemize}
        \end{itemize}
        
        \vspace{0.3cm}
        
        \textbf{Matriz de Confusão:}
        \begin{table}
            \footnotesize
            \begin{tabular}{cc|cc}
                \multicolumn{2}{c}{} & \multicolumn{2}{c}{\textbf{Predito}} \\
                \multicolumn{2}{c|}{} & Normal & TB \\
                \cline{2-4}
                \multirow{2}{*}{\rotatebox{90}{\textbf{Real}}} & Normal & 46 & 4 \\
                & TB & 5 & 45 \\
            \end{tabular}
        \end{table}
        
        \column{0.5\textwidth}
        \textbf{Interpretação Clínica:}
        \begin{block}{Pontos Fortes}
            \begin{itemize}
                \item ✓ Alta sensibilidade (90\%)
                \item ✓ Alta especificidade (92\%)
                \item ✓ Balanceamento adequado
                \item ✓ AUC-ROC excelente (96\%)
            \end{itemize}
        \end{block}
        
        \begin{alertblock}{Atenção}
            \begin{itemize}
                \item 5 falsos negativos (FN)
                \item 4 falsos positivos (FP)
                \item Sempre confirmar com exames adicionais
            \end{itemize}
        \end{alertblock}
    \end{columns}
\end{frame}

\begin{frame}{Comparação Visual dos Modelos}
    \begin{figure}
        \centering
        \includegraphics[width=0.9\textwidth]{imgs/model_comparison.png}
        \caption{Comparação de métricas entre os três modelos pré-treinados}
    \end{figure}
\end{frame}
\begin{frame}{Tabela Comparativa Detalhada}
    \begin{figure}
        \centering
        \includegraphics[width=0.9\textwidth]{imgs/model_comparison_table.png}
        \caption{Tabela comparativa completa dos modelos}
    \end{figure}
\end{frame}


%===========================================
% 5. CONCLUSÕES
%===========================================
\section{Conclusões}

\begin{frame}{Conclusões}
    \begin{block}{Principais Contribuições}
        \begin{itemize}
            \item Implementação de transfer learning com feature extraction
            \item Avaliação de ResNet-50 para detecção de TB
            \item Resultados superiores à média da literatura
            \item Pipeline reproduzível com Docker
        \end{itemize}
    \end{block}
    
    \vspace{0.3cm}
    
    \begin{block}{Lições Aprendidas}
        \begin{itemize}
            \item Transfer learning com backbone congelado é essencial para datasets pequenos
            \item Feature extraction evita overfitting (vs fine-tuning completo)
            \item Data augmentation crucial para generalização
            \item Balanceamento entre sensibilidade e especificidade é alcançável
        \end{itemize}
    \end{block}
\end{frame}


\begin{frame}{Referências}
    \footnotesize
    \begin{thebibliography}{99}
        \bibitem{lakhani2017}
        Lakhani, P., \& Sundaram, B. (2017). Deep learning at chest radiography: automated classification of pulmonary tuberculosis by using convolutional neural networks. \textit{Radiology}, 284(2), 574-582.
        
        \bibitem{rajaraman2018}
        Rajaraman, S., et al. (2018). Pre-trained convolutional neural networks as feature extractors toward improved malaria parasite detection in thin blood smear images. \textit{PeerJ}, 6, e4568.
        
        \bibitem{hwang2016}
        Hwang, S., et al. (2016). A novel approach for tuberculosis screening based on deep convolutional neural networks. \textit{Medical Imaging 2016: Computer-Aided Diagnosis}.
        
        \bibitem{pan2010}
        Pan, S. J., \& Yang, Q. (2010). A survey on transfer learning. \textit{IEEE Transactions on Knowledge and Data Engineering}, 22(10), 1345-1359.
        
        \bibitem{he2016}
        He, K., et al. (2016). Deep residual learning for image recognition. \textit{Proceedings of the IEEE Conference on Computer Vision and Pattern Recognition}, 770-778.
    \end{thebibliography}
\end{frame}

\begin{frame}[standout]
    \Huge Obrigado!
    
    \vspace{1cm}
    
    \Large Perguntas?
    
    \vspace{1cm}
    
\end{frame}

\end{document}
