\documentclass{beamer}
\usepackage[utf8]{inputenc}
\usepackage[T1]{fontenc}
\usepackage{amsmath, amssymb}
\usepackage{graphicx}
\usepackage{url}
\usepackage{hyperref}
\usepackage{tikz}
\usepackage[backend=biber,style=ieee]{biblatex}
\graphicspath{{./images/}}
\addbibresource{references.bib}
\usetheme{Madrid}
\usecolortheme{dolphin}

\setbeamertemplate{navigation symbols}{}
\setbeamertemplate{caption}{\raggedright\insertcaption\par}
\setbeamertemplate{headline}{
    \begin{tikzpicture}[remember picture,overlay]
        \node[anchor=north east, xshift=-0.3cm, yshift=-0.2cm]
            at (current page.north east)
            {\includegraphics[width=1.5cm]{ufrj_logo_horizontal.png}};
    \end{tikzpicture}
}

\title{Uma investigação comparativa entre dados sintéticos vs aumento de dados tradicional em problemas de classificação de imagens}
\author{João Vítor Correia Pessoa}
\institute{Universidade Federal do Rio de Janeiro (UFRJ)}

\makeatletter
\setbeamertemplate{footline}
{
  \leavevmode%
  \hbox{%
  \begin{beamercolorbox}[wd=\paperwidth,ht=3ex,dp=1.5ex]{date in head/foot}%
    \hspace*{1ex}
    \usebeamerfont{date in head/foot}
    \hfill
    \insertshortdate{}%
    \hspace*{2em}%
    \insertframenumber{} / \inserttotalframenumber\hspace*{1ex}%
  \end{beamercolorbox}}%
  \vskip0pt
}
\makeatother

\begin{document}
% TITLEPAGE
\begin{frame}
    \titlepage
\end{frame}

% INTRODUCTION
\section{Motivação}

\begin{frame}{Motivação}
\begin{itemize}
    \item Muitos domínios possuem poucas amostras rotuladas
    \item Técnicas convencionais de aumentos de dados podem ser insuficientes
    \item GANs modernos, como StyleGAN2-ADA, permitem gerar imagens sintéticas realistas
    \item Objetivo: avaliar impacto do uso de dados sintéticos vs. aumento tradicional no treinamento de classificadores
\end{itemize}
\end{frame}

\section{Arquitetura}

\begin{frame}{StyleGAN2-ADA}
\begin{itemize}
    \item Modelo generativo baseado em StyleGAN2 com foco em datasets pequenos
    \item ADA (Adaptive Discriminator Augmentation): aplica aumento adaptativo para evitar overfitting
    \item Permite treinar com poucas centenas de imagens mantendo qualidade
    \item Usado aqui para gerar imagens sintéticas que expandem o conjunto de dados real
\end{itemize}
\end{frame}

\begin{frame}{Arquitetura geral da StyleGAN2-ADA}
   \begin{figure} 
    \centering
    \includegraphics[width=0.50\textwidth]{images/stylegan2_ada_arch.png}
    \caption{Arquitetura geral do modelo StyleGAN2-ADA}
  \end{figure}
\end{frame}

\begin{frame}{ADA}
   \begin{figure} 
    \centering
    \includegraphics[width=0.80\textwidth]{images/da.png}
    \caption{Tipos de aumento de imagens realizados pelo algoritmo ADA}
  \end{figure}
\end{frame}

\begin{frame}{ResNet50}
Introduz \textbf{blocos residuais} com conexões de atalho (\textit{skip connections}) que permitem:
\begin{itemize}
    \item Treinamento de redes muito profundas
    \item Redução da degradação de performance à medida que profundidade aumenta
\end{itemize}
\begin{figure} 
    \centering
    \includegraphics[width=0.70\textwidth]{images/resnet50_arch.png}
    \caption{Arquitetura geral do modelo ResNet50}
\end{figure}
\end{frame}

\begin{frame}{EfficientNet-B0}
Baseada em blocos \textbf{MBConv} (Mobile Inverted Bottleneck), eficientes e leves, possui com características principais:
\begin{itemize}
    \item Alto desempenho com baixo custo computacional
    \item Ótima escolha para datasets pequenos e treinamento rápido
    \item Favorece generalização devido ao design eficiente
\end{itemize}
\begin{figure} 
    \centering
    \includegraphics[width=0.60\textwidth]{images/efficientNet_B0_arch.png}
    \caption{Arquitetura geral do modelo EfficientNet-B0}
\end{figure}
\end{frame}

\section{Experimentos}

\begin{frame}{Parâmetros importantes}
\begin{figure} 
    \centering
    \includegraphics[width=0.60\textwidth]{images/table1.png}
\end{figure}
\begin{figure} 
    \centering
    \includegraphics[width=0.60\textwidth]{images/table2.png}
\end{figure}
\end{frame}

\begin{frame}{Parâmetros importantes}
\begin{figure} 
    \centering
    \includegraphics[width=0.60\textwidth]{images/table3.png}
\end{figure}
\begin{figure} 
    \centering
    \includegraphics[width=0.60\textwidth]{images/table4.png}
\end{figure}
\end{frame}

\begin{frame}{Parâmetros importantes}
\begin{figure} 
    \centering
    \includegraphics[width=0.60\textwidth]{images/table5.png}
\end{figure}
\begin{figure} 
    \centering
    \includegraphics[width=0.60\textwidth]{images/table6.png}
\end{figure}
\end{frame}

\begin{frame}{Geração das imagens}
\begin{figure} 
    \centering
    \includegraphics[width=0.75\textwidth]{images/real_images.png}
    \caption{Imagens reais}
\end{figure}
\begin{figure} 
    \centering
    \includegraphics[width=0.75\textwidth]{images/gen_images.png}
    \caption{Imagens geradas}
\end{figure}
\end{frame}

\section{Resultados}

\begin{frame}{Resumo dos experimentos}
\begin{figure} 
    \centering
    \includegraphics[width=0.60\textwidth]{images/table7.png}
\end{figure}
\begin{figure} 
    \centering
    \includegraphics[width=0.90\textwidth]{images/table8.png}
\end{figure}
\end{frame}

\begin{frame}{Curvas de Acurácia}
    \begin{columns}[T]
        \begin{column}{0.48\textwidth}
            \centering
            \includegraphics[width=\textwidth]{images/comparativo_resnet50_acc.png}
        \end{column}
        \begin{column}{0.48\textwidth}
            \centering
            \includegraphics[width=\textwidth]{images/comparativo_efficientnet_b0_acc.png}
        \end{column}
    \end{columns}
\end{frame}

\begin{frame}{Curvas de Perda}
    \begin{columns}[T]
    \begin{column}{0.48\textwidth}
        \centering
        \includegraphics[width=\textwidth]{images/comparativo_resnet50_loss.png}
    \end{column}
    \begin{column}{0.48\textwidth}
        \centering
        \includegraphics[width=\textwidth]{images/comparativo_efficientnet_b0_loss.png}
    \end{column}
\end{columns}
\end{frame}

\begin{frame}{Matrizes de confusão}
    \begin{columns}[T]
        \begin{column}{0.25\textwidth}
            \centering
            \includegraphics[width=\textwidth, trim=20 20 20 20, clip]{images/confusion_resnet50_generated_only.png}
        \end{column}
        
        \begin{column}{0.25\textwidth}
            \centering
            \includegraphics[width=\textwidth, trim=20 20 20 20, clip]{images/confusion_resnet50_real_aug_standard.png}
        \end{column}
        
        \begin{column}{0.25\textwidth}
            \centering
            \includegraphics[width=\textwidth, trim=20 20 20 20, clip]{images/confusion_resnet50_real_only.png}
        \end{column}
        
        \begin{column}{0.25\textwidth}
            \centering
            \includegraphics[width=\textwidth, trim=20 20 20 20, clip]{images/confusion_resnet50_real_plus_gan.png}
        \end{column}
    \end{columns}

    \vspace{0.3cm}

    \begin{columns}[T]
        \begin{column}{0.25\textwidth}
            \centering
            \includegraphics[width=\textwidth, trim=20 20 20 20, clip]{images/confusion_efficientnet_b0_generated_only.png}
        \end{column}

        \begin{column}{0.25\textwidth}
            \centering
            \includegraphics[width=\textwidth, trim=20 20 20 20, clip]{images/confusion_efficientnet_b0_real_aug_standard.png}
        \end{column}

        \begin{column}{0.25\textwidth}
            \centering
            \includegraphics[width=\textwidth, trim=20 20 20 20, clip]{images/confusion_efficientnet_b0_real_only.png}
        \end{column}

        \begin{column}{0.25\textwidth}
            \centering
            \includegraphics[width=\textwidth, trim=20 20 20 20, clip]{images/confusion_efficientnet_b0_real_plus_gan.png}
        \end{column}
    \end{columns}
\end{frame}

\begin{frame}{Análise dos resultados}
\begin{itemize}
    \item Os dados sintéticos contribuíram para reduzir overfitting nos modelos profundos
    \item O treinamento apenas com dados sintéticos apresentou melhor desempenho na rede EfficientNet-B0
    \item Para o conjunto de dados analisado, o aumento de dados tradicional continua relevante
\end{itemize}
\end{frame}

\section{Considerações finais}

\begin{frame}{Considerações finais}
\begin{itemize}
    \item Modelos GAN avançados como StyleGAN2-ADA se mostram eficazes em alguns cenários de poucos dados
    \item Dados sintéticos podem complementar a augmentation tradicional
    \item As melhorias dependem da qualidade das imagens geradas
    \item Trabalhos futuros: 
    \begin{itemize}
        \item Avaliar datasets em que augmentation tradicional não gera impacto significativo
        \item Avaliar a relação entre quantidade máxima de dados sintéticos no treinamento com a qualidade do gerador
    \end{itemize}
\end{itemize}
\end{frame}

\end{document}