\documentclass{beamer}
\usepackage{amsfonts,amsmath,oldgerm}
\usetheme{sintef}
\usepackage{tabularx,booktabs}
\usepackage{tikz}
\usepackage[most]{tcolorbox}
\usetikzlibrary{shapes,arrows,positioning, fit}

\definecolor{normalColor}{RGB}{102,194,165}     % verde-azulado
\definecolor{transientColor}{RGB}{252,141,98}   % laranja
\definecolor{faultColor}{RGB}{141,160,203}      % azul
\definecolor{normalClassColor}{RGB}{171,221,164}
\definecolor{faultClassColor}{RGB}{253,174,97}



\newcommand{\testcolor}[1]{\colorbox{#1}{\textcolor{#1}{test}}~\texttt{#1}}

\usefonttheme[onlymath]{serif}

\titlebackground*{assets/background}

% adicionar o numero na lista final da apresentação
\setbeamertemplate{bibliography item}{\insertbiblabel}

\newcommand{\hrefcol}[2]{\textcolor{cyan}{\href{#1}{#2}}}

\title{Comparação de Modelos de Deep Learning para Detecção de Anomalias em Poços de Petróleo
}
\course{CPE 727 - Deep Learning}
\author{Ana Clara Loureiro Cruz}
\IDnumber{%
  \href{mailto:anaclaralcruz@poli.ufrj.br}{anaclaralcruz@poli.ufrj.br}
}

\begin{document}

\maketitle

\section{Introdução}

\begin{frame}{Descrição do problema}

  \begin{itemize}
    \item Grandes volumes de dados de processo na indústria de óleo e gás (pressão, vazão, temperatura, etc.)
    \item Necessidade de monitoramento dos dados
    \item Dificuldade de identificar anomalias no processo
          
  \end{itemize}
  \begin{block}{Questão central}
            Modelos de \textit{deep learning} podem melhorar a detecção precoce de
            anomalias em poços de petróleo?
          \end{block}
\end{frame}


%------------------------------------------------
\begin{frame}{Motivação}

  \begin{itemize}
    \item Falhas em poços e linhas de produção podem resultar em:
          \begin{itemize}
            \item perdas significativas de produção;
            \item aumento de custos de manutenção corretiva;
            \item riscos à segurança operacional e ao meio ambiente.
          \end{itemize}
    \item Modelos de aprendizado de máquina clássicos podem não capturar adequadamente padrões temporais complexos e relações não lineares entre sensores.
  \end{itemize}

\end{frame}

%------------------------------------------------
\begin{frame}{Objetivo}

\begin{itemize}
  \item Selecionar e implementar 3 arquiteturas de \textit{deep learning}
        adequadas à modelagem de séries temporais de poços.
  \item Comparar os resultados e propor melhorias para os sistemas de detecção de anomalias em séries temporais.
\end{itemize}

\end{frame}


\section{Revisão bibliográfica}

\begin{frame}{Métodos propostos}

\small

\textbf{RF com atributos estatísticos \cite{Marins2021}}
\begin{itemize}
    \item Método: extração de estatísticas em janelas deslizantes das séries de poços,
          normalização e classificador \textit{Random Forest} (RF) para detecção e
          classificação de falhas na base 3W \cite{Vargas2019}.
    \item Resultado: \textit{balanced accuracy} de aproximadamente $94{,}2\%$ no
          classificador multiclasses.\,
\end{itemize}

\medskip

\textbf{SWT + arquitetura modular \cite{Dias2024}}
\begin{itemize}
    \item Método: proposta de um sistema modular e extração de atributos
          baseados em decomposição \textit{stationary wavelet transform} (SWT),
          combinados a atributos estatísticos, para classificar eventos da base 3W.
    \item Resultado: \textit{balanced accuracy} de $98{,}6\%$ no problema
          multiclasses.
\end{itemize}

\medskip


\end{frame}


\begin{frame}{ThreeWToolkit \cite{toolkit}}
    \begin{itemize}
        \item Pacote em Python 3 para experimentos com o 3W Dataset.
        \item Focado em detecção e classificação de eventos indesejáveis em poços offshore.
        \item Ferramentas:
          \begin{itemize}
            \item Dataloaders para 3W Dataset;
            \item Pre-processors for 3W Dataset;
            \item Modelo MLP configurável;
          \end{itemize}
    \end{itemize}
\end{frame}


\section{Metodologia}

\begin{frame}{Descrição do Dataset}
\begin{itemize}
    \item Classificação de séries temporais multivariadas
    \item Séries temporais anotadas divididas em 10 classes
\end{itemize}

No escopo deste trabalho: subset de 3 classes
Janelamento de passo 1 e tamanho 100

\begin{figure}[h]
    \centering
    \includegraphics[width=0.5\textwidth]{imagens/serie_temporal.png} 
    \caption{Exemplo de série temporal anotada (Imagem retirada de \cite{Marins2021}).}
    \label{fig:serie_temporal}
\end{figure}
\end{frame}

\begin{frame}{Estratégia de comparação}

\begin{columns}[T] % alinhar topo (Top)
    \begin{column}{0.54\textwidth}
        \textbf{Modelos escolhidos}
        \begin{itemize}
            \item Multi-layer Perceptron (MLP) implementado em \cite{toolkit} (baseline)
            \item MLP com regularização (dropout)
            \item Convolutional Neural Network (CNN) \cite{cnn}
            \item Temporal Convolutional Network (TCN) \cite{tcn}
        \end{itemize}
    \end{column}

    \begin{column}{0.44\textwidth}
        \textbf{Métricas de avaliação}
        \begin{itemize}
            \item Acurácia
            \item Precisão
            \item Recall
            \item F1-Score
            \item Tempo de execução
        \end{itemize}
    \end{column}
\end{columns}

\end{frame}


\begin{frame}{Padronização das pipelines}
\begin{block}{Hiperparâmetros de treinamento \cite{toolkit}}
\begin{itemize}
    \item Otimizador: Adam
    \item Função de perda: Cross-entropy
    \item Taxa de aprendizado: 0{,}0005
    \item Batch size: 64
    \item Cross-validation: ativado
    \item 70\% treino 15\% teste 15\% validação
\end{itemize}
\end{block}

\end{frame}

\begin{frame}{Padronização das pipelines}
\begin{block}{Pré-processamento \cite{toolkit}}
\begin{itemize}
    \item Imputação de dados faltantes:
    \begin{itemize}
        \item Estratégia: média (\textit{mean})
        \item Aplicado a todas as colunas
    \end{itemize}
    \item Normalização:
    \begin{itemize}
        \item Tipo: norma L2
    \end{itemize}
    \item Janela temporal:
    \begin{itemize}
        \item Tamanho da janela: 100 amostras
    \end{itemize}
\end{itemize}
\end{block}
\end{frame}

\begin{frame}{Hiperparâmetros dos Modelos}

\begin{block}{MLP (com e sem regularização)}
    Dados de entrada: Extração de dados estatísticos da janela temporal 
\begin{itemize}
    \item Tipo: MLP totalmente conectado
    \item Camadas escondidas: (128, 64, 32)
    \item Tamanho da saída: 3 classes (0, 1, 2)
    \item Função de ativação: ReLU
    \color{red} {\item Regularização: L2}
\end{itemize}

\end{block}

\end{frame}

\begin{frame}{Hiperparâmetros dos Modelos}

\begin{block}{CNN 1D e TCN}
    \begin{itemize}
    \item Número de canais de entrada: 1
    \item Filtros por camada: (32, 64, 128)
    \item Tamanho do kernel: 3
    \item Função de ativação: ReLU
    \item Taxa de dropout: 0.1
\end{itemize}

\end{block}

\end{frame}

\section{Resultados}

\begin{frame}{Comparação dos modelos}

\begin{table}[ht]
\centering
\begin{tabular}{lccccc}
\hline
Modelo & Acurácia balanceada & Precisão & Recall & F1 & Tempo (s) \\
\hline
MLP puro & 0.9395 & 0.9743 & 0.9744 & 0.9742 & 1479\\
MLP com regularização & 0.9413 & 0.9733 & 0.9733 & 0.9733 & 1513 \\
CNN & 0.8824 &  0.9539 & 0.9541 & 0.9538 & 1654 \\
TCN & 0.9147 & 0.9884 & 0.9472 & 0.9474 & 605
\end{tabular}
\end{table}
\end{frame}

\begin{frame}{Exemplos de evolução da função-custo}
\begin{figure}[h!]
    \centering

    \begin{minipage}{0.48\textwidth}
        \centering
        \includegraphics[width=\linewidth]{imagens/MLP_sem_regularizacao.png}
        \caption{Exemplo de evolução da função-custo no MLP puro}
        \label{fig:imagem1}
    \end{minipage}
    \hfill
    \begin{minipage}{0.48\textwidth}
        \centering
        \includegraphics[width=\linewidth]{imagens/MLP_com_regularizacao.png}
        \caption{Exemplo de evolução da função-custo no MLP com dropout}
        \label{fig:imagem2}
    \end{minipage}
\end{figure}

\end{frame}


\begin{frame}{Exemplos de evolução da função-custo}
\begin{figure}[h!]
    \centering

    \begin{minipage}{0.48\textwidth}
        \centering
        \includegraphics[width=\linewidth]{imagens/CNN.png}
        \caption{Exemplo de evolução da função-custo na CNN}
        \label{fig:imagem1}
    \end{minipage}
    \hfill
    \begin{minipage}{0.48\textwidth}
        \centering
        \includegraphics[width=\linewidth]{imagens/TCN.png}
        \caption{Exemplo de evolução da função-custo na TCN}
        \label{fig:imagem2}
    \end{minipage}
\end{figure}

\end{frame}



\section{Conclusão}

\begin{frame}{Discussão dos resultados}
\begin{itemize}
    \item Os MLP baseado em características estatísticas não foi superado pelos modelos convolucionais
    \item A CNN talvez se beneficiaria de mais poder computacional
    \item A TCN mostrou bons resultados no trade-of entre desempenho e custo
    \item Aumentar o subset para obter melhores resultados
\end{itemize}
\end{frame}


\section{Referências} 
\begin{frame}[allowframebreaks]
        \frametitle{Bibliography} 
        \bibliographystyle{ieeetr}
        \bibliography{assets/presentation_bib.bib}
\end{frame}


\backmatter
\end{document}
