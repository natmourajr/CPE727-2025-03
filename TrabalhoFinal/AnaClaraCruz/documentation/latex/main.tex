\documentclass{beamer}
\usepackage{amsfonts,amsmath,oldgerm}
\usetheme{sintef}
\usepackage{tabularx,booktabs}
\usepackage{tikz}
\usepackage[most]{tcolorbox}
\usetikzlibrary{shapes,arrows,positioning}

\newcommand{\testcolor}[1]{\colorbox{#1}{\textcolor{#1}{test}}~\texttt{#1}}

\usefonttheme[onlymath]{serif}

\titlebackground*{assets/background}

% adicionar o numero na lista final da apresentação
\setbeamertemplate{bibliography item}{\insertbiblabel}

\newcommand{\hrefcol}[2]{\textcolor{cyan}{\href{#1}{#2}}}

\title{Comparação de Modelos de Deep Learning para Detecção Precoce de Anomalias em Poços de Petróleo
}
\course{CPE 727 - Deep Learning}
\author{Ana Clara Loureiro Cruz}
\IDnumber{%
  \href{mailto:anaclaralcruz@poli.ufrj.br}{anaclaralcruz@poli.ufrj.br}
}

\begin{document}
\maketitle

\section{Introdução}


\begin{frame}{Problema a ser abordado}
%Natureza das anomalias (gradual vs súbita)

%Desafios: sensores ruidosos, variabilidade operacional, volume de dados

%limitacao do dataset

% por que eu quero comparar??

%Limitações dos métodos tradicionais
\end{frame}

\begin{frame}{Motivação}
%Por que detectar anomalias precocemente em poços?

%Impactos em segurança, produção e integridade do ativo
\end{frame}

\begin{frame}{Objetivo}
\end{frame}


\section{Revisão bibliográfica}

\begin{frame}{RandomForest}
\end{frame}

\begin{frame}{Random Forest2}
\end{frame}

\begin{frame}{Toolkit}
\end{frame}

\begin{frame}{O que está faltando?}
\end{frame}

\section{Método proposto}

\begin{frame}{Visão Geral do Método}
Pipeline end-to-end: preprocessamento → janela → modelo → decisão
\end{frame}

\begin{frame}{Descrição do Dataset}
Origem das séries (3W ou equivalente)

Tipos de anomalia

Distribuição das classes

subset escolhido
\end{frame}

\begin{frame}{Pré-processamento}
Imputação

Normalização

Janela deslizante

Balanceamento / Estratificação

\end{frame}

\begin{frame}{Modelos Avaliados}
MLP (baseline)

TCN

CNN 1D

LSTM / GRU

\end{frame}

\begin{frame}{Métricas de Avaliação}
Balanced Accuracy

Recall para eventos raros

Detecção precoce (quantos segundos/janelas antes)

Custo computacional

\end{frame}

\begin{frame}{Estratégia de Treinamento}
Validação cruzada ou hold-out

Hiperparâmetros

Critérios de parada
\end{frame}

\section{Resultados}

\begin{frame}{Comparação dos modelos}
Slide 18 — Desempenho Quantitativo dos Modelos

Tabela com métricas principais

Comparação direta

\end{frame}

\section{Conclusão}

\begin{frame}{Discussão dos resultados}
Modelos mais fortes

Modelos mais leves

Trade-offs entre desempenho e custo
\end{frame}
\begin{frame}{Limitações e escalabilidade}
usar o dataset inteiro
\end{frame}


\section{Referências} 
\begin{frame}[allowframebreaks]
        \frametitle{Bibliography} 
        \bibliographystyle{ieeetr}
        \bibliography{assets/presentation_bib.bib}
\end{frame}


\backmatter
\end{document}
