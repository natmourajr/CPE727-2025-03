\documentclass{beamer}
\usepackage[portuguese]{babel}
\usepackage{amsfonts,amsmath,oldgerm}
\usepackage{url}
\usepackage{hyperref}
\usepackage{pgfgantt}
\usepackage{algorithm}
\usepackage{algpseudocode}
\usepackage{float} % Pacote essencial para a opção [H]
\usepackage{tabularx}
\usepackage{booktabs}   % Para linhas de tabela profissionais
\usepackage[table]{xcolor} % Para cores em tabelas

\definecolor{marker_color}{RGB}{77,201,232}
\definecolor{marker_done_color}{RGB}{45,115,133}
\definecolor{finish_color}{RGB}{232,166,60}

\newcommand{\ganttchartyear}{\number\year}
\newcommand{\ganttchartmonth}{\ifnum\month<10 0\fi\number\month} 

\usetheme{sintef}

\newcommand{\testcolor}[1]{\colorbox{#1}{\textcolor{#1}{test}}~\texttt{#1}}
\usefonttheme[onlymath]{serif}

\titlebackground*{assets/background}

% adicionar o numero na lista final da apresentação
\setbeamertemplate{bibliography item}{\insertbiblabel}

\newcommand{\hrefcol}[2]{\textcolor{cyan}{\href{#1}{#2}}}

\title{Convolutional Neural Networks}
% \subtitle{Resolvendo o Problema de Picasso \cite{gliozzi2022combining}}
\course{CPE 727 - Aprendizado Profundo}
\author{Brenno Rodrigues de Carvalho da Silva, Gabriel Guimarães, Lucas Alexandre, Raphael Nagao}

\IDnumber{\href{mailto:brennorcs@poli.ufrj.br}{brennorcs@poli.ufrj.br}, \href{mailto:guimaraes921@poli.ufrj.br}{guimaraes921@poli.ufrj.br}, \href{mailto:lucas.alexandre@coppe.ufrj.br}{lucas.alexandre@coppe.ufrj.br}, \href{mailto:rnagao@cos.ufrj.br}{rnagao@cos.ufrj.br}}

\begin{document}
\maketitle

\section{Revisão: Redes Convolucionais (CNNs)}

\footlinecolor{}
\begin{frame}{O que são CNNs?}
\begin{itemize}
    \item CNNs (Convolutional Neural Networks) são redes neurais projetadas para processar dados com estrutura espacial, como imagens.
\end{itemize}
\begin{figure}
    \centering
    \includegraphics[width=0.5\linewidth]{Convolutional layer.png}
    \caption{Camadas das CNN \cite{geron2022hands}.}
    \label{fig:placeholder}
\end{figure}
\end{frame}

\begin{frame}{O que são CNNs?}
\begin{itemize}
    \item Compostas por:
    \begin{itemize}
        \item Camadas \textbf{convolucionais} (extração de padrões locais)
        \item Camadas de \textbf{pooling} (redução de dimensionalidade)
        \item Camadas densas (classificação)
    \end{itemize}
\end{itemize}
\begin{figure}
    \centering
    \includegraphics[width=0.5\linewidth]{Typical CNN architecture.png}
    \caption{Arquitetura típica de CNN \cite{geron2022hands}.}
\end{figure}
\end{frame}

\begin{frame}{Camadas Convolucionais}
\begin{itemize}
    \item Aplicam \textbf{filtros} (kernels) sobre a imagem para gerar mapas de ativação.
    \item Cada filtro aprende a detectar um padrão específico (bordas, texturas, etc).
    \item Resulta em mapas de ativação (feature maps).
\end{itemize}
\begin{figure}
         \centering
         \includegraphics[width=0.43\linewidth]{Applying 2 different filters.png}
         \caption{Aplicando 2 filtros diferentes para obter 2 mapas de características. \cite{geron2022hands}.}
         \label{fig:placeholder}
     \end{figure}
\end{frame}

\begin{frame}{Pooling e suas limitações}
\begin{itemize}%[<alert@2>]
    \item Reduz as dimensões dos mapas de ativação
    \item Os dois tipos mais comuns:
    \begin{itemize}
        \item Max pooling: obtém o valor máximo da região.
        \item Average Pooling: obtém a média dos valores da região. 
    \end{itemize}
\end{itemize}

\begin{columns}
\begin{column}{0.85\textwidth}
\begin{figure}
    \centering
    \includegraphics[width=0.6\linewidth]{Max pooling layer.png}
    \caption{Camada de \textit{max pooling} (\textit{kernel pooling} de 2x2, stride de 2, sem uso de \cite{geron2022hands}.}
\end{figure}
\end{column}
\begin{column}{0.15\textwidth}
\begin{colorblock}[yellow]{sintefred}{Problema:}
perde informações espaciais importantes.
\small
\end{colorblock}
\end{column}
\end{columns}
\end{frame}


\section{A matemática por trás}

\begin{frame}{O que é a Convolução Discreta}
Em redes neurais tradicionais (totalmente conectadas), geralmente "achatamos" os dados de entrada, como uma imagem, em um único vetor longo. Isso ignora uma propriedade fundamental desses dados: a estrutura intrínseca. Imagens, por exemplo, têm eixos (altura, largura) onde a ordem importa, e canais (como R, G, B) que representam diferentes "visões" dos dados.
\end{frame}

\begin{frame}{O que é a Convolução Discreta}
A convolução discreta é uma transformação linear projetada especificamente para preservar e explorar essa noção de ordem.

Diferente de uma transformação afim (multiplicação por matriz densa), a convolução é:
\begin{itemize}
    \item \textbf{Esparsa:} Apenas algumas unidades de entrada contribuem para uma determinada unidade de saída.
    \item \textbf{Reutiliza parâmetros:} Os mesmos pesos (o "kernel") são aplicados em múltiplos locais da entrada.
\end{itemize}
\end{frame}

\begin{frame}{O Mecanismo da Convolução}
O processo funciona da seguinte maneira:
\begin{enumerate}
    \item \textbf{Mapa de Características de Entrada (Input):} A entrada da camada (por exemplo, uma imagem).
    \item \textbf{Kernel (Filtro):} Um pequeno conjunto de pesos (ex: 3x3) que "desliza" sobre o mapa de entrada.
    \item \textbf{Operação:} Em cada posição, calculamos o produto elemento a elemento entre o kernel e a porção da entrada que ele sobrepõe.
    \item \textbf{Soma:} Os resultados desses produtos são somados para gerar um único valor de saída.
    \item \textbf{Mapa de Características de Saída (Output):} Esse valor se torna um "pixel" no mapa de saída. O processo se repete até o kernel percorrer toda a entrada, formando um novo mapa.
\end{enumerate}
\end{frame}




\begin{frame}{O Mecanismo da Convolução}
    \begin{figure} % Apenas UM ambiente figure para os dois gráficos
        
        \begin{minipage}{0.15\textwidth} % Define a primeira coluna (com 48% da largura do texto)
            \centering
            \includegraphics[width=\linewidth]{figures/kernel.png}
            \caption{Kernel \cite{dumoulin2016guide}}
            \label{fig:Kernel} % Label corrigido para ser único
        \end{minipage}
        \hfill % Adiciona um espaço flexível entre as duas colunas
        \begin{minipage}{0.6\textwidth} % Define a segunda coluna (com 48% da largura do texto)
            \centering
            % Dica: É uma boa prática usar nomes de arquivo sem acentos (ex: roc_curve_Validacao.png)
            \includegraphics[width=\linewidth]{figures/Convolução.png}
            \caption{Convolução 2D \cite{dumoulin2016guide}}
            \label{fig:conv} % Label corrigido para ser único
        \end{minipage}

    \end{figure}
\end{frame}

\begin{frame}{A Aritmética da Convolução}
Como prever o tamanho da saída de uma camada convolucional?
\vspace{0.5cm}

O tamanho da saída ( $o$ ) é determinado por quatro hiperparâmetros:
\begin{itemize}
    \item $i$ (Input Size): O tamanho (altura ou largura) da entrada;
    \item $k$ (Kernel Size): O tamanho (altura ou largura) do kernel;
    \item $s$ (Stride): O "passo", ou seja, a distância que o kernel se move entre operações;
    \item $p$ (Zero Padding): O número de zeros adicionados nas bordas da entrada.
\end{itemize}
\end{frame}

\begin{frame}{Caso 1: Sem Padding ($p=0$), Strides Unitários ($s=1$)}
Este é o caso mais básico. O kernel ( $k$ ) desliza pela entrada ( $i$ ) um pixel de cada vez.
\vspace{0.5cm}
\begin{equation}
    o = (i - k) + 1
    \label{eq:0_padding}
\end{equation}
Exemplo (Baseado na Fig \ref{fig:0_padding} ):
\begin{itemize}
    \item Input $i = 4$ (uma matriz 4x4)
    \item Kernel $k = 3$ (uma matriz 3x3)
    \item Stride $s = 1$,
    \item Padding $p = 0$
    \item Saída $o = (4 - 3) + 1 = 2$.
\end{itemize}

\end{frame}

\begin{frame}{Caso 1: Sem Padding ($p=0$), Strides Unitários ($s=1$)}
\begin{figure}
    \centering
    \includegraphics[width=0.6\linewidth]{figures/Sem_paddig.png}
    \caption{Convolução Kernel 3x3 com uma entrada 4x4 sem padding \cite{dumoulin2016guide}}
    \label{fig:0_padding}
\end{figure}
\end{frame}

\begin{frame}{Caso 2: Padding ($p>0$), Strides Unitários ($s=1$)}
O padding adiciona zeros ao redor da entrada. Se adicionarmos $p$ zeros de cada lado, o "tamanho efetivo" da entrada se torna $i + 2p$. A fórmula geral é ajustada:
\vspace{0.5cm}
\begin{equation}
    o = (i +2p - k) + 1
    \label{eq:padding>0}
\end{equation}
Exemplo (Baseado na Fig \ref{fig:2_padding} ):
\begin{itemize}
    \item Input $i = 5$ (uma matriz 5x5)
    \item Kernel $k = 4$ (uma matriz 4x4)
    \item Stride $s = 1$,
    \item Padding $p = 2$
    \item Saída $o = (5 +2\cdot2 -4) + 1 = 6$.
\end{itemize}
\begin{equation}
    o = (i +2p - k) + 1
    \label{eq:exemplo1}
\end{equation}

\end{frame}

\begin{frame}{Caso 2: Padding ($p=2$), Strides Unitários ($s=1$)}
\begin{figure}
    \centering
    \includegraphics[width=0.6\linewidth]{figures/2_padding.png}
    \caption{Convolução Kernel 4x4 com uma entrada 5x5 com  padding = 2 \cite{dumoulin2016guide}}
    \label{fig:2_padding}
\end{figure}
\end{frame}

\begin{frame}{Caso 3: Half padding Strides Unitários ($s=1$)}
Uma propriedade muito desejável é fazer com que a saída tenha o mesmo tamanho da entrada ($o = i$). Isso é comum em arquiteturas modernas (como ResNets) para manter as dimensões espaciais.
\vspace{0.5cm}
Isso é alcançado (quando $s=1$ e $k$ é ímpar ($k = 2n+1$)) definindo o padding como: $p = \lfloor k / 2 \rfloor$
\begin{equation}
    o = i +2(k/2) - (k - 1) = i + 2n -2n = i
    \label{eq:half_padding}
\end{equation}

\end{frame}

\begin{frame}{Caso 3: Half padding Strides Unitários ($s=1$)}

    Exemplo (Baseado na Fig \ref{fig:half_padding} ):
    \begin{itemize}
        \item Input $i = 5$ (uma matriz 5x5)
        \item Kernel $k = 3$ (uma matriz 3x3)
        \item Stride $s = 1$,
        \item Padding $p = \lfloor 3 / 2 \rfloor = 1$.
    \begin{equation}
        o = (5 + 2 \cdot 1 - 3) + 1 = (5 + 2 - 3) + 1 = 5
        \label{eq:exemplo3}
    \end{equation}
\end{itemize}

\end{frame}


\begin{frame}{Caso 3: Half padding Strides Unitários ($s=1$)}
    \begin{figure}
        \centering
        \includegraphics[width=0.6\linewidth]{figures/half_padding.png}
        \caption{Convolução Kernel 3x3 com uma entrada 5x5 com  stride = 1 e half padding \cite{dumoulin2016guide}}
        \label{fig:half_padding}
    \end{figure}
\end{frame}

\begin{frame}{Caso 4: Sem Padding ($p=0$), Strides Não Unitários ($s > 1$)}
    Quando o stride (passo) é maior que 1 ($s>1$), o kernel "pula" pela entrada. Isso funciona como uma forma de subamostragem (downsampling).
    \vspace{0.5cm}
    Como o kernel dá passos de tamanho $s$, o número de posições possíveis diminui. 
        \begin{equation}
            o = \left\lfloor \frac{i - k}{s} \right\rfloor + 1
            \label{eq:stride+}
        \end{equation}

\end{frame}

\begin{frame}{Caso 4: Sem Padding ($p=0$), Strides Não Unitários ($s > 1$)}

    Exemplo (Baseado na Fig \ref{fig:half_padding} ):
    \begin{itemize}
        \item Input $i = 5$ (uma matriz 5x5)
        \item Kernel $k = 3$ (uma matriz 3x3)
        \item Stride $s = 2$,
        \item Padding $p = 0$.
        \item $\lfloor \cdot \rfloor$ representa a função piso.
    \begin{equation}
        o =\lfloor \frac{5 -3}{2}\rfloor + 1 = 2
        \label{eq:exemplo4}
    \end{equation}
    \end{itemize}

\end{frame}


\begin{frame}{Caso 4: Sem Padding ($p=0$), Strides Não Unitários ($s > 1$)}
\begin{figure}

    \centering
    \includegraphics[width=0.6\linewidth]{figures/arbitrary_stride.png}
    \caption{Convolução Kernel 3x3 com uma entrada 5x5 com  stride = 2 e padding = 0 \cite{dumoulin2016guide}}
    \label{fig:half_padding}
\end{figure}
\end{frame}

\begin{frame}{Caso Geral}
    Finalmente, podemos combinar todos os parâmetros (padding e strides não unitários) para obter a relação mais geral.
    \vspace{0.5cm}
    \begin{equation}
        o = \left\lfloor \frac{i +2p - k}{s} \right\rfloor + 1
        \label{eq:general}
    \end{equation}
\end{frame}

\section{Aritmética do Pooling}

\begin{frame}{Pooling}
\vspace{0.5cm}
O \textit{pooling} é uma operação utilizada para reduzir a dimensionalidade espacial dos mapas de ativação, com objetivo de diminuir o custo computacional.

\begin{figure}
    \centering
    \includegraphics[width=0.4\linewidth]{figures/pooling.png}
    \caption{Exemplo de camada de \textit{pooling}}
    \cite{akhtar2020interpretation}
    \label{fig:pooling}
\end{figure}
\end{frame}

\begin{frame}{Tipos de Pooling}
As camadas de pooling resumem as informações presentes em pequenas regiões da imagem.

Os tipos mais comuns são:
\begin{enumerate}
    \item \textbf{Max Pooling:} Seleciona o maior valor dentro da região analisada.
    Preserva as características mais intensas (bordas, texturas fortes).
    \item \textbf{Average Pooling:} Calcula a média dos valores dentro da região.
    Produz uma representação mais suave, útil quando se deseja manter uma noção geral da região.
\end{enumerate}
\end{frame}


\begin{frame}{Parâmetros que determinam o tamanho da saída}
    O tamanho da saída de uma camada de \textit{pooling} depende de três valores:

    \begin{itemize}
        \item \textbf{Input Size}: tamanho (altura ou largura) da entrada;
        \item (\textbf{Kernel Size}): tamanho da janela de \textit{}{pooling};
        \item (\textbf{Stride}): distância que a janela se desloca entre cada operação.
    \end{itemize}

    \noindent
    Consideraremos \textbf{padding = 0} (sem adição de bordas).
    
\end{frame}

\begin{frame}{Fórmula}
    A dimensão de saída ($o$) é dada por:

\begin{equation}
    o = \left\lfloor \frac{i - k}{s} \right\rfloor + 1
    \label{eq:stride+}
    \end{equation}
    
    \noindent
    Onde:
    \begin{itemize}
        \item $i$ é o tamanho da entrada;
        \item $k$ é o tamanho da janela;
        \item $s$ é o passo de deslizamento (\textit{stride});
        \item $\lfloor \cdot \rfloor$ representa a função piso.
    \end{itemize}
    
    Essa fórmula é aplicada separadamente para altura e largura.
\end{frame}

\begin{frame}{Exemplos}
    \begin{figure}
    \centering
    \includegraphics[width=0.4\linewidth]{figures/exemplos-pooling.png}
    \caption{Exemplos de \textit{Max Pool} e \textit{Average Pool}} 
    \cite{ahamed2020handwritten}
    \label{fig:pooling}
\end{figure}
\end{frame}

\section{Convolução Transposta}

\begin{frame}{O que é uma Convolução Transposta?}

A convolução transposta (ou DCNN) é a operação inversa que permite que as redes neurais aumentem o tamanho dos dados, transformando uma entrada de baixa dimensão em uma saída de alta dimensão, o oposto do que uma convolução padrão faz.\cite{dumoulin2016guide}

\begin{figure}
    \centering
    \includegraphics[width=0.5\linewidth]{figures/Conv_transposta/Example_transp_conv.png}
    \caption{Exemplo de convolução transposta \cite{DOSREIS2024113258}}
    \label{fig:transp_conv_example}
\end{figure}
    
\end{frame}


\begin{frame}{Objetivo}
\begin{columns}[T,totalwidth=\linewidth]
    \begin{column}{0.48\linewidth}
    \vspace{1.0cm}
        \begin{itemize}
            \item Reverter a Convolução
            \item Projeção para Maior Dimensão
            \item Decodificação e Reconstrução
            \item Resolução de Tarefas de Baixo Nível
            \item Expansão da Dimensão de Características
        \end{itemize}        
    \end{column}

    \begin{column}{0.52\linewidth}
        \begin{figure}
            \centering
            \includegraphics[width=\linewidth]{figures/Conv_transposta/deconv.png}
            \caption{Exemplo de deconvolução numa imagem real \cite{ong2015fullyconvolutionalnetworkssemantic}}
            \label{fig:deconv}
        \end{figure}
    \end{column}    
\end{columns}    
\end{frame}

\begin{frame}[t]{Matemática da Deconvolução}
\vspace{1.0cm}
Dimensões de saída \cite{PytorchTransp} 

\begin{equation}
    H_{out} = (H_{in} - 1) \times \text{s}[0] - 2 \times \text{p}[0] + \text{d}[0] \times (\text{k}[0] - 1) + \text{p'}[0] + 1
\end{equation}

\begin{equation}
    W_{out} = (W_{in} - 1) \times \text{s}[1] - 2 \times \text{p}[1] + \text{d}[1] \times (\text{k}[1] - 1) + \text{p'}[1] + 1
\end{equation}

\end{frame}


\begin{frame}{Algorítimo}

\begin{columns}[T,totalwidth=\linewidth]

    % ======== Coluna ESQUERDA (tópicos) ========
    \begin{column}{0.50\linewidth}
        \begin{itemize}
            \item A \textbf{Conv2D} aplica um filtro (kernel) sobre a imagem,
                  extraindo padrões locais.
            \item O \textbf{stride} e o \textbf{padding} controlam o tamanho da saída.
            \item O \textbf{ConvTranspose2D} faz a operação inversa,
                  reconstruindo a forma espacial original.
            \item Ambas usam o mesmo kernel no exemplo ao lado.
        \end{itemize}
    \end{column}

    % ======== Coluna DIREITA (imagem) ========
    \begin{column}{0.50\linewidth}
        \vspace*{-2.0cm}
        \begin{figure}
            \centering
            \includegraphics[width=0.4\linewidth]{figures/Conv_transposta/conv_transposta_criada.png}
            \caption{Convolução transposta gerada no python}
            \label{fig:conv_transpose_example}
        \end{figure}
    \end{column}

\end{columns}
    
\end{frame}

\begin{frame}{No zero padding, unit strides, transposed}

\begin{figure}[H]
    \centering
    \includegraphics[width=0.8\linewidth]{figures/Conv_transposta/No_zero_unit_strides_transposed.png}
    \caption{Convolução transposta (i' = 2, k' = 3, s' = 1, p' = 2)\cite{dumoulin2016guide}}
    \label{fig:nozero_unitstride}
\end{figure}

\begin{equation}
    o' = i' + (k - 1)
\end{equation}
    
\end{frame}


\begin{frame}{Zero padding, unit strides, transposed}

\begin{figure}[H]
    \centering
    \includegraphics[width=0.8\linewidth]{figures/Conv_transposta/zero_pading_unit_strides.png}
    \caption{Convolução transposta (i' = 6, k' = 4, s' = 1, p' = 1)\cite{dumoulin2016guide}}
    \label{fig:nozero_unitstride}
\end{figure}

\begin{equation}
    o' = i' + (k - 1) - 2p
\end{equation}
    
\end{frame}


\begin{frame}{No zero padding, non-unit strides, transposed}
\begin{figure}[H]
    \centering
    \includegraphics[width=0.8\linewidth]{figures/Conv_transposta/No_zero_non_unit_strides.png}
    \caption{Convolução transposta (i' = 2, \~{i}' = 3, k' = 3, s' = 1, p' = 2)\cite{dumoulin2016guide}}
    \label{fig:nozero_unitstride}
\end{figure}

\begin{equation}
    o' = s(i' - 1) + k
\end{equation}
\end{frame}


\begin{frame}{Zero padding, non-unit strides, transposed}
\begin{figure}[H]
    \centering
    \includegraphics[width=0.8\linewidth]{figures/Conv_transposta/zer_pading_non_unit_strides.png}
    \caption{Convolução transposta (i' = 3, \~{i}' = 5, k' = 3, s' = 1, p' = 1)\cite{dumoulin2016guide}}
    \label{fig:nozero_unitstride}
\end{figure}

\begin{equation}
    o' = s(i' - 1) + k - 2p
\end{equation}
\end{frame}

\section{Modelos Alternativos}

\begin{frame}{LeNet-5 (1998)}
\begin{figure}
    \centering
    \includegraphics[width=1\linewidth]{figures/modelos/Arquitetura_LeNet5.jpg}
    \caption{Arquitetura da LeNet-5 \cite{lenet_lecun}.}
    \label{fig:placeholder}
\end{figure}
\end{frame}

\begin{frame}{AlexNet (2012)}
\begin{itemize}
    \item Considerada um marco no deep learning modernos;
    \item Utilização de grande volume de dados para o treinamento (ImageNet);
    \item Uso de GPU;
    \item Dropout;
    \item Data Augmentation;
    \item Substituição da tanh() pela ReLu().
\end{itemize}
\end{frame}

\begin{frame}{AlexNet (2012)}
\begin{figure}
    \centering
    \includegraphics[width=1\linewidth]{figures/modelos/Arquitetura_AlexNet.png}
    \caption{Arquitetura da LeNet-5 \cite{alexnet}.}
    \label{fig:placeholder}
\end{figure}
\end{frame}

\begin{frame}{VGGNet (2014)}
\begin{itemize}
    \item Aumentar a performance, aumentando a profundidade da rede;
    \item Pode-se aproximar filtros largos como 5x5 e 7x7 utilizando filtros 3x3:
    \begin{itemize}
        \item um filtro 5x5 tem o mesmo campo receptivo de 2 filtros 3x3;
        \item um filtro 7x7 tem o mesmo campo receptivo de 3 filtros 3x3.
    \end{itemize}
    \item Diminuição do número de parâmetros;
    \item Adiciona mais camadas não lineares ReLu com o uso de filtros menores.
\end{itemize}
\end{frame}

\begin{frame}{VGGNet (2014)}
\begin{itemize}
\begin{figure}
    \centering
    \includegraphics[width=0.43\linewidth]{figures/modelos/familia_vggnet.jpg}
    \caption{Família de modelos VGGNet \cite{vggnet}.}
    \label{fig:placeholder}
\end{figure}
\end{itemize}
\end{frame}

\begin{frame}{GoogLeNet / Inception (2012)}
\begin{itemize}
    \item Busca de aumentar o desempenho da AlexNet sem aumento de custo computacional;
    \item Criação do Módulo Inception;
    \item Introdução da camada de bottlenecks (convoluções de 1x1).
\end{itemize}
\end{frame}

\begin{frame}{GoogLeNet / Inception (2012)}
\begin{figure}
    \centering
    \includegraphics[width=1\linewidth]{figures/modelos/inception.png}
    \caption{Módulo Inception \cite{googlenet}.}
    \label{fig:placeholder}
\end{figure}
\end{frame}

\begin{frame}{GoogLeNet / Inception (2012)}
\begin{figure}
    \centering
    \includegraphics[width=1\linewidth]{figures/modelos/arquitetura_googlenet.png}
    \caption{Arquitetura completa da GoogLeNet \cite{googlenet}.}
    \label{fig:placeholder}
\end{figure}
\end{frame}

\begin{frame}{ResNet (2015)}
\begin{columns}
\begin{column}{0.6\textwidth}
\begin{itemize}
    \item Problema de degradação para redes muito profundas;
    \item Prova experimental, utilizando camadas de identidade, que redes mais profuntas devem ser pelo menos tão boas quanto suas versões menores;
    \item Criação de blocos de resíduos (shortcut connection).
    \begin{itemize}
        \item Algoritmos de resíduos são mais performáticos em visão computacional (VLAD, Fisher Vectors, Multigrid)
        \item Resíduo converge mais rápido e é mais fácil de otimizar
    \end{itemize}
\end{itemize}
\end{column}

\begin{column}{0.4\textwidth}
\small
\begin{figure}
    \centering
    \includegraphics[width=0.85\linewidth]{figures/modelos/residual_block.png}
    \caption{Bloco de Resíduo (shortcut connection) \cite{resnet}.}
    \label{fig:placeholder}
\end{figure}
\end{column}
\end{columns}
\end{frame}

\begin{frame}{ResNet (2015)}
\begin{figure}
    \centering
    \includegraphics[width=0.9\linewidth]{figures/modelos/arquitetura_resnet.png}
    \caption{Arquitetura família ResNet \cite{resnet}.}
    \label{fig:placeholder}
\end{figure}
\end{frame}

\begin{frame}{DenseNet (2017)}
\begin{itemize}
    \item Busca superar a limitação da ResNet de overfit para mais de centenas de camadas;
    \item Shortcut connection entre todas as camadas, com concatenação dos resultados dos mapas de características ao invés de somar;
    \item A preservação das características aprendidas faz com que haja um melhor fluxo das informações e do gradiente no treinamento.
\end{itemize}
\end{frame}

\begin{frame}{DenseNet (2017)}
\begin{figure}
    \centering
    \includegraphics[width=0.9\linewidth]{figures/modelos/arquitetura_densenet.png}
    \caption{Arquitetura DenseNet \cite{densenet}.}
    \label{fig:placeholder}
\end{figure}
\end{frame}

\begin{frame}{CapsNet (2017)}
\begin{columns}
\begin{column}{0.4\textwidth}
\small
Uma cápsula é um grupo de neurônios cuja saída é um vetor:
\begin{itemize}
    \item \textbf{Comprimento do vetor}: probabilidade de existência.
    \item \textbf{Direção do vetor}: propriedades da entidade (posição, orientação, etc).
\end{itemize}
\end{column}
\begin{column}{0.6\textwidth}
\small
\begin{figure}
    \centering
    \includegraphics[width=.7\linewidth]{figures/modelos/Cápsulas.png}
    \caption{dinâmica de cápsulas “filho” e “pai”, com vetores de ativação e previsão (ĉ‑vetores) conectando-os.}
    \label{fig:capsulas}
\end{figure}
\end{column}
\end{columns}
\end{frame}

\begin{frame}{CapsNet (2017)}

\begin{columns}
\begin{column}{0.3\textwidth}
\small
\begin{itemize}
    \item Camada convolucional inicial.
    \item \textbf{Primary Capsules} (lower-level): vetores derivados de filtros convolucionais.
    \item \textbf{Digit Capsules} (higher-level): uma por classe (ex: dígitos 0-9).
    \item Classificação baseada na magnitude do vetor de saída.
\end{itemize}
\end{column}

\begin{column}{0.7\textwidth}
\begin{figure}
    \centering
    \includegraphics[width=1\linewidth]{figures/modelos/CapsNetArch.png}
    \caption{Arquitetura da CapsNet \cite{mukhometzianov2018capsnet}}
    \label{fig:placeholder}
\end{figure}
\end{column}
\end{columns}

\end{frame}

\begin{frame}{EfficientNet (2019)}
\begin{columns}
\begin{column}{0.6\textwidth}
\begin{itemize}
    \item Ao contrário dos outros modelos até então a EfficientNet não busca introduzir uma inovação na arquitetura das CNNs, mas sim numa maneira de escalar e crescer a rede;
    \item \textbf{Compound Scalling (depth, width, resolution):} ao invés de escalar a rede arbitrariamente em uma das 3 dimensões, a EfficientNet propõe um método para escalar uniformemente largura, profundidade e resolução da rede utilizando um conjunto fixo de coeficientes de escalonamento.
\end{itemize}
\end{column}

\begin{column}{0.4\textwidth}
\small
\begin{figure}
    \centering
    \includegraphics[width=0.85\linewidth]{figures/modelos/otimizacao_parametros_efficientnet.jpg}
    \caption{Coeficientes de escalonamento do compound scalling \cite{efficientnet}.}
    \label{fig:placeholder}
\end{figure}
\end{column}
\end{columns}
\end{frame}

\begin{frame}{EfficientNet (2019)}
\begin{figure}
    \centering
    \includegraphics[width=0.5\linewidth]{figures/modelos/comparativo_efficientnet.jpg}
    \caption{Comparativo tamanho e precisão entre modelos no ImageNet \cite{efficientnet}.}
    \label{fig:placeholder}
\end{figure}
\end{frame}

% \begin{frame}{\textit{Matrix Capsules with EM Routing} \cite{hinton2018matrix} }

% \begin{itemize}
%     \item Representação das Cápsulas:
%      \begin{itemize}
%          \item Ao invés de vetores agora as capsulas são matrizes que representam a pose;
%      \end{itemize}
%     \item Mecanismo de Roteamento: 
%     \begin{itemize}
%          \item Algoritmo iteratvo \textit{Expectation-Maximization (EM)};
%      \end{itemize}
%      \item Função de ativação: 
%     \begin{itemize}
%          \item Função logística;
%      \end{itemize}
%      \item Número de Parâmetros:
%          \begin{itemize}
%          \item  Número de Parâmetros: Conseguiu uma redução significativa de parâmetros em comparação com o CapsNet original.
%      \end{itemize}
% \end{itemize}
% \end{frame}

% \begin{frame}{\textit{Efficient-CapsNet} \cite{mazzia2021efficient}}

% \begin{itemize}
%     \item Representação das Cápsulas:
%      \begin{itemize}
%          \item Retorna à ideia origial onde as capsulas são vetores;
%      \end{itemize}
%     \item Mecanismo de Roteamento: 
%          \begin{itemize}
%          \item Algoritmo \textit{Self-attention routing} (não iterativo);
%      \end{itemize}
%     \item Função de ativação: 
%          \begin{itemize}
%          \item Utiliza uma variante da função \textit{squash} original;
%      \end{itemize}
%      \item Número de Parâmetros:
%          \begin{itemize}
%          \item  Número de Parâmetros: Conseguiu uma redução significativa de parâmetros em comparação com o CapsNet original e o Matrix CapsNet.
%      \end{itemize}     
% \end{itemize}
% \end{frame}

% \begin{frame}{\textit{GRU + CapsNet} \cite{grucaps}}
% \begin{columns}

% \begin{column}{0.55\textwidth}
% \begin{itemize}
%     \item Explora a tarefa de identificar emoções em tweets NLP (dataset: WASSA 2018);    

%     \item Arquitetura: 
%     \begin{itemize}
%          \item Embedding layer (word2vec), Bi-GRU layer com \textbf{capsule network}.
%          \item Features de GRU são passados para a Caps Net;
%      \end{itemize}
     
%     \item Função de ativação e Mecanismo de Roteamento: 
%      \begin{itemize}
%          \item Função \textit{squash} original e \textit{dynamic routing}.
%      \end{itemize}
     
% \end{itemize}
% \end{column}

% \begin{column}{0.40\textwidth}
%  \begin{figure}
%     \centering
%     \includegraphics[width=\linewidth]{assets/grucaps.png}
%     \caption{Arquitetura GRU + CapsNet \cite{grucaps}.}
% \end{figure}
% \end{column}

% \end{columns}
% \end{frame}


% \begin{frame}{\textit{TimeCaps} \cite{timecaps}}
% \begin{columns}
% \begin{column}{0.55\textwidth}
% \begin{itemize}
%     \item Tarefa de \textbf{classificação} e reconstrução de sinais ECG.    
    
%     \item Arquitetura: 
%          \begin{itemize}
%          \item Convolutional Layer com Primary Capsule Networks.
%          \item Uma camada recebe features de tempo e a outra de espaço. 
%      \end{itemize}
     
%     \item Função de ativação e Mecanismo de Roteamento: 
%          \begin{itemize}
%          \item Variação do \textit{squash} adaptado, utilizando \textit{dynamic routing}.
%      \end{itemize}
     
% \end{itemize}
% \end{column}

% \begin{column}{0.36\textwidth}
%  \begin{figure}
%     \centering
%     \includegraphics[width=\linewidth]{assets/timecaps.png}
%     \caption{Arquitetura TimeCaps \cite{timecaps}.}  
% \end{figure}
% \end{column}
% \end{columns}

% \end{frame}


\section{Conclusões}

% \begin{frame}[fragile]{Comparação Profissional}
%     % Use um tamanho de fonte menor para garantir que tudo caiba confortavelmente
%     \footnotesize

%     \textbf{Comparativo entre arquiteturas de Redes Capsulares (CapsNet)}
    
%     % \rowcolors{<linha inicial>}{<cor da linha ímpar>}{<cor da linha par>}
%     % Usamos uma cor cinza muito clara (gray!10) para as linhas pares.
%     \rowcolors{2}{}{gray!10}

%     % Use tabularx para ajustar a tabela à largura do texto do slide
%     \begin{tabularx}{\textwidth}{@{} l X X X X X @{}} % O @{} remove o espaço extra nas bordas
%         \toprule
%         \textbf{Característica} & \textbf{CapsNet} & \textbf{\textit{Matrix Capsules}} & \textbf{\textit{Efficient-CapsNet}} & \textbf{GRU + CapsNet} & \textbf{TimeCaps} \\
%         \midrule

%         \textbf{Roteamento} & 
%         \textit{Dynamic Routing}, iterativo & 
%         \textit{Expectation-Maximization Routing} &  % , iterativo (baseado em clusters Gaussianos)
%         \textit{Self-Attention}, não-iterativo & 
%         \textit{Dynamic Routing} & 
%         \textit{Dynamic Routing} \\
%         \addlinespace

%         \textbf{Representação da Pose} & 
%         Vetores (orientação do vetor)  & 
%         Matrizes 4x4 & 
%         Vetores (orientação do vetor) & 
%         Vetores (orientação do vetor) & 
%         Vetores (orientação do vetor)\\
%         \addlinespace

%         \textbf{Representação da Presença} & 
%         Comprimento do vetor  & 
%         Unidade logística separada  & 
%         Comprimento do vetor & 
%         Comprimento do vetor & 
%         Comprimento do vetor\\
%         \addlinespace

%         \textbf{Função de Ativação} & 
%         \textit{Squash} & 
%         Logística & 
%         Variante da \textit{squash} & 
%         \textit{Squash} & 
%         Variante da \textit{squash} \\

%         \bottomrule
%     \end{tabularx}
% \end{frame}

\begin{frame}{Vantagens das CNNs}
    \begin{columns}
    \begin{column}{0.5\textwidth}
    \begin{itemize}
    \item \textbf{Redução de dimensionalidade.}
        
    \item \textbf{Prevenção de overfitting.}
    
    \item \textbf{Preservação das informações mais relevantes.}
    
    \item \textbf{Compatibilidade com arquiteturas profundas.} 
\end{itemize}
    \end{column}
    \begin{column}{0.5\textwidth}
    \begin{figure}
        \centering
        \includegraphics[width=0.8\linewidth]{figures/maxpooled_1-1.png}
        \caption{Redução da dimensionalidade.}
    \end{figure} 
    \end{column}
    \end{columns}
\end{frame}

\begin{frame}{Limitações das CNNs}
    \begin{columns}
    \begin{column}{0.5\textwidth}
    \begin{itemize}
        \item Podem não capturar relações espaciais entre partes do objeto.
        \item Sensíveis a rotações, escala e perspectiva ("\textit{Picasso problem}").
        \item Dependem de grandes volumes de dados e \textit{data augmentation}.
        \item Pooling remove informação estrutural importante.
        \end{itemize}
    \end{column}
    \begin{column}{0.5\textwidth}
    \begin{figure}
        \centering
        \includegraphics[width=0.8\linewidth]{DATA AUGMENTATION.png}
        \caption{DATA AUGMENTATION - novas instâncias de treinamento a  partir de instâncias existentes. \cite{geron2022hands}}
    \end{figure} 
    \end{column}
    \end{columns}
\end{frame}

\section{Referências Bibliográficas} 
\begin{frame}[allowframebreaks]
        \frametitle{Referências Bibliográficas} 
        \bibliographystyle{ieeetr}
        \bibliography{presentation_bib.bib}
\end{frame}

\backmatter
\end{document}
