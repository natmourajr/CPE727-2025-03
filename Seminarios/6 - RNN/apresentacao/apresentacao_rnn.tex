\documentclass[10pt]{beamer}

% ======================
% Pacotes básicos
% ======================
\usepackage[utf8]{inputenc}
\usepackage[T1]{fontenc}
\usepackage[provide=*,main=portuguese]{babel}
\usepackage{amsmath, amsfonts, amssymb}
\usepackage{graphicx}
\usepackage{booktabs}
\usepackage{xcolor}
\usepackage{array}
\usepackage{tabularx}
\usepackage[numbers]{natbib} 

\usetheme{Madrid} % pode trocar por: Berlin, CambridgeUS, Metropolis (se instalado), etc.

% ======================
% Infos da capa
% ======================
\title[RNNs em Séries Temporais]{RNNs em Séries Temporais:\\
Arquiteturas, Treinamento e Aplicações}
\author{Fernanda Villa Verde, Leonardo Britto, Luiza Helena e Rodrigo Petrus}
\institute{Modelagem de Séries Temporais com RNNs - LSTM / GRU / BiLSTM / BiGRU}
\date{\today}

% ======================
% Início
% ======================
\begin{document}

\begin{frame}
  \titlepage
\end{frame}

% ======================
% Agenda
% ======================
\begin{frame}{Agenda}
  \tableofcontents
\end{frame}

% ======================
\section{Motivação e Contexto Histórico}
% ======================
% ======================
\begin{frame}{Motivação e Contexto Histórico}
\footnotesize

\textbf{Por que RNNs?}  
As redes feedforward assumem entradas independentes e processam cada amostra isoladamente.  
Entretanto, muitas tarefas reais exigem \textbf{modelagem temporal}:
\begin{itemize}
  \item Séries financeiras e sensoriais
  \item Áudio e fala
  \item Texto e linguagem natural
  \item Sequências de vídeo
\end{itemize}

\medskip
\textbf{Avanço conceitual:}  
As \textbf{Redes Neurais Recorrentes (RNNs)} introduziram a noção de \textbf{memória de estados anteriores},  
permitindo representar dependências temporais ao longo da sequência.


\vspace{0.3cm}

    \centering
    \includegraphics[width=0.4\linewidth]{/home/ferna/CPE727-2025-03/Seminarios/6 - RNN/apresentacao/figs/RNN.png}
    \vspace{0.2cm}
    \scriptsize
    \textit{RNN: fluxo temporal de estados com memória.}


\end{frame}


% ======================
\section{Linha do Tempo das RNNs}
% ======================
\begin{frame}{Linha do Tempo das RNNs}
\small
\setlength{\tabcolsep}{3pt} % (opcional) deixa a tabela mais compacta

\begin{center}
\begin{tabularx}{\linewidth}{|c|l|>{\raggedright\arraybackslash}X|}
\hline
\textbf{Ano} & \textbf{Autor(es)} & \textbf{Contribuição} \\ \hline

1982   & Hopfield                & Rede recorrente como memória associativa (não sequencial). \\ \hline
1986   & Jordan                  & Recorrência via \textbf{saída} anterior. \\ \hline
1990   & Elman                   & Recorrência via \textbf{estado oculto} (RNN padrão). \\ \hline
1990   & Werbos                  & Formaliza \textbf{BPTT} para treinar RNNs. \\ \hline
1991   & Hochreiter              & Demonstra o \textbf{vanishing gradient}. \\ \hline
1997   & Hochreiter \& Schmidhuber & Introduzem \textbf{LSTM} (estado de célula preserva gradiente). \\ \hline
1997   & Schuster \& Paliwal     & Introduzem \textbf{BiRNN} (contexto passado + futuro). \\ \hline
2005  & Graves \& Schmidhuber         & Combinação: \textbf{LSTM + BiRNN}. \\ \hline
2014   & Cho et al.              & Introduzem \textbf{GRU} (simplifica LSTM). \\ \hline
2014  & Chung, J.          & Combinação: \textbf{GRU + BiRNN}. \\ \hline

\end{tabularx}
\end{center}
\end{frame}


% ======================
\section{Modelos RNN Clássicos}
% ======================


% =========================================================
% SLIDE 1 — Hopfield (1982)
% =========================================================

% ======================
\begin{frame}{Hopfield Network (1982) - Memória associativa \cite{hopfield1982}}
\footnotesize

\textbf{Ideia central:}  
A Hopfield Network é uma rede recorrente totalmente conectada e simétrica,  
capaz de armazenar e recuperar padrões — funcionando como uma \textbf{memória associativa}.

\vspace{0.3cm}
\centering
\includegraphics[width=0.65\linewidth]{/home/ferna/CPE727-2025-03/Seminarios/6 - RNN/apresentacao/figs/hopfield.png}

\vspace{0.15cm}
\scriptsize
\textit{Rede totalmente conectada: cada neurônio recebe sinais dos demais.}

\vspace{0.3cm}
\footnotesize
\textbf{Resumo conceitual:}
\begin{itemize}
  \item \textbf{Conexões simétricas:} garantem estabilidade e convergência.
  \item \textbf{Atualização recorrente:} cada neurônio depende do estado dos outros.
  \item \textbf{Mínimo de energia:} estado fixo da rede corresponde a uma memória recuperada.
  \item \textbf{Padrões aprendidos:} configurações estáveis codificadas nos pesos.
\end{itemize}

\vspace{0.2cm}
\scriptsize
\textit{A Hopfield Network introduz o princípio de memória associativa — base conceitual das RNNs.}

\end{frame}
% ======================



% =========================================================
% SLIDE 2 — Jordan Network (1986)
% =========================================================
\begin{frame}{Jordan Network (1986) \cite{jordan1986}}
\footnotesize

\textbf{Ideia central:} memória via \textbf{feedback da saída} em \emph{context units}.

\medskip
\textbf{Contexto (copia saída anterior):}
\[
\mathbf{c}_t \;=\; \mathbf{y}_{t-1}
\]

\textbf{Estado oculto:}
\[
\mathbf{h}_t \;=\; \phi\!\left(W_{xh}\,\mathbf{x}_t \;+\; W_{ch}\,\mathbf{c}_t \;+\; \mathbf{b}_h\right)
\]

\textbf{Saída:}
\[
\mathbf{y}_t \;=\; \psi\!\left(W_{hy}\,\mathbf{h}_t \;+\; \mathbf{b}_y\right)
\]

\medskip
\textbf{Onde:} \(\phi\) tipicamente \(\tanh\) (ou ReLU) e \(\psi\) pode ser \(\text{softmax}\) (classificação) ou identidade (regressão).

\medskip
\textbf{Resumo:} a \textbf{memória} vem de \(\mathbf{y}_{t-1}\) (não de \(\mathbf{h}_{t-1}\)).
\end{frame}

% =========================================================
% SLIDE 3 — Elman / SRN (1990)
% =========================================================
\begin{frame}{Elman (1990) - RNN: Simple Recurrent Network (SRN) \cite{elman1990}}
\footnotesize

\textbf{Ideia central:} recorrência no \textbf{estado oculto} — a RNN “padrão”.

\medskip
\textbf{Estado oculto (recorrente):}
\[
\mathbf{h}_t \;=\; \phi\!\left(W_{xh}\,\mathbf{x}_t \;+\; W_{hh}\,\mathbf{h}_{t-1} \;+\; \mathbf{b}_h\right)
\]

\textbf{Saída:}
\[
\mathbf{y}_t \;=\; \psi\!\left(W_{hy}\,\mathbf{h}_t \;+\; \mathbf{b}_y\right)
\]

 \textbf{Notação:}
  \begin{itemize}
    \item $x_t$: entrada no tempo $t$
    \item $h_t$: estado oculto (memória)
    \item $y_t$: saída predita
    \item $W_{hh}$: pesos de transição (estado $\to$ estado)
    \item $W_{xh}$: pesos de entrada (entrada $\to$ estado)
    \item $W_{hy}$: pesos de saída (estado $\to$ saída)
    \item $b_h, b_y$: vieses
  \end{itemize}

\medskip
\textbf{Inicialização típica:} \(\mathbf{h}_0=\mathbf{0}\) (ou parâmetro treinável).

\textbf{Observações:}
\begin{itemize}
  \item \(\phi\) costuma ser \(\tanh\); \(\psi\) pode ser \(\text{softmax}\) (classificação).
  \item Base para variantes modernas (LSTM/GRU) que mitigam \emph{vanishing/exploding gradients}.
\end{itemize}
\end{frame}


\begin{frame}{Werbos (1990) — Backpropagation Through Time (BPTT) \cite{werbos1990}}
\footnotesize

\textbf{Forward:}
\[
h_t = \phi(W_{xh} x_t + W_{hh} h_{t-1} + b_h), 
\qquad
y_t = \psi(W_{hy} h_t + b_y)
\]

\medskip
\textbf{Perda total:}
\[
L = \sum_{t=1}^{T} \ell(y_t, \tilde{y}_t)
\]

\medskip
\textbf{BPTT — o gradiente retropropaga no tempo:}
\[
\frac{\partial L}{\partial h_t}
=
\frac{\partial L}{\partial y_t}\frac{\partial y_t}{\partial h_t}
+
\frac{\partial L}{\partial h_{t+1}}
\frac{\partial h_{t+1}}{\partial h_t}
\]

\[
\frac{\partial h_{t+1}}{\partial h_t} = D_{t+1} W_{hh},
\quad
D_{t+1} = \mathrm{diag}(1 - \tanh^2(\cdot))
\]

\[
\frac{\partial L}{\partial W_{hh}}
=
\sum_{t=1}^{T}
\left(\frac{\partial L}{\partial h_t}\right)(h_{t-1})^\top
\]

\medskip
\textbf{Ponto-chave:}  
O erro “viaja para trás” através do tempo, acumulando dependências.
\end{frame}


\begin{frame}{Werbos (1990) — Backpropagation Through Time (BPTT) \cite{werbos1990} (cont.)}
\footnotesize
\begin{columns}[T,onlytextwidth]

  % ----- Coluna da figura -----
  \begin{column}{0.50\textwidth}
    \centering
    \includegraphics[width=\linewidth]{/home/ferna/CPE727-2025-03/Seminarios/6 - RNN/apresentacao/figs/BPTT.png} % ajuste o caminho se necessário
    \scriptsize
    \emph{A RNN vista ao longo do tempo. Cada estado $s_t$ influencia o próximo.}
  \end{column}

  % ----- Coluna da intuição -----
  \begin{column}{0.50\textwidth}

    \textbf{Ideia central}
    \begin{itemize}\itemsep4pt
      \item A RNN processa a sequência passo a passo.
      \item Cada estado oculto $h_t$ guarda uma \textbf{memória} do passado.
      \item O passado influencia o futuro através da recorrência.
    \end{itemize}

    \medskip

    \textbf{Por que isso é importante?}
    \begin{itemize}\itemsep4pt
      \item Permite modelar dependências temporais.
      \item Útil para texto, fala, séries temporais, sensores, controle, etc.
    \end{itemize}

    \medskip

    \textbf{Resumo mental:}
    \scriptsize
    \[
    \text{estado novo} = f(\text{entrada atual},\
    \text{lembrança do passado})
    \]

  \end{column}

\end{columns}
\end{frame}


\begin{frame}{Hochreiter (1991) — Vanishing Gradient \cite{hochreiter1991}}
\footnotesize

\textbf{Equação-chave (forma simplificada do gradiente):}
\[
\frac{\partial L}{\partial h_t}
=
\frac{\partial L}{\partial h_T}
\prod_{k=t+1}^{T}
\frac{\partial h_k}{\partial h_{k-1}}
\]

\textbf{Para uma RNN simples com $\tanh$:}
\[
\frac{\partial h_k}{\partial h_{k-1}}
=
D_k \, W_{hh},
\quad\text{onde}\quad
D_k = \mathrm{diag}(1 - \tanh^2(\cdot)),
\;\; \|D_k\|_2 \le 1.
\]

\textbf{Consequência:}
\[
\left\|\frac{\partial h_k}{\partial h_{k-1}}\right\|_2 \le \|W_{hh}\|_2
\;\;\Rightarrow\;\;
\left\|\frac{\partial L}{\partial h_t}\right\|
\;\lesssim\;
\|W_{hh}\|_2^{\,T-t}
\]

\medskip
\textbf{Interpretação:}
\begin{itemize}
  \item Se $\|W_{hh}\|_2 < 1$: o gradiente \textbf{decai exponencialmente} → \textbf{vanishing gradient}.
  \item Se $\|W_{hh}\|_2 > 1$: o gradiente pode \textbf{explodir}.
  \item Esse efeito \textbf{não é de implementação}, mas da \textbf{matemática da recorrência}.
\end{itemize}

\end{frame}

\begin{frame}{Hochreiter \& Schmidhuber (1997) — LSTM \cite{hochreiter1997}}
\footnotesize

\textbf{Ideia central:} introduzir um \textbf{estado de célula} $c_t$ com \textbf{rota de gradiente quase constante},
permitindo \textbf{memória de longo prazo} e evitando \textbf{vanishing gradient}.

\medskip
\textbf{Equações da LSTM:}
\[
\begin{aligned}
\mathbf{i}_t &= \sigma(W_{xi}\mathbf{x}_t + W_{hi}\mathbf{h}_{t-1} + \mathbf{b}_i) & \text{(porta de entrada)}\\
\mathbf{f}_t &= \sigma(W_{xf}\mathbf{x}_t + W_{hf}\mathbf{h}_{t-1} + \mathbf{b}_f) & \text{(porta de esquecimento)}\\
\tilde{\mathbf{c}}_t &= \tanh(W_{xc}\mathbf{x}_t + W_{hc}\mathbf{h}_{t-1} + \mathbf{b}_c) & \text{(conteúdo candidato)}\\[4pt]
\mathbf{c}_t &= \mathbf{f}_t \odot \mathbf{c}_{t-1} + \mathbf{i}_t \odot \tilde{\mathbf{c}}_t & \text{(estado de célula)}\\[4pt]
\mathbf{o}_t &= \sigma(W_{xo}\mathbf{x}_t + W_{ho}\mathbf{h}_{t-1} + \mathbf{b}_o) & \text{(porta de saída)}\\
\mathbf{h}_t &= \mathbf{o}_t \odot \tanh(\mathbf{c}_t) & \text{(estado oculto)}
\end{aligned}
\]

\medskip
\textbf{Por que isso evita vanishing gradient?}

\[
\frac{\partial \mathbf{c}_t}{\partial \mathbf{c}_{t-1}} = \mathbf{f}_t
\]

Se $\mathbf{f}_t \approx 1$, então:
\[
\frac{\partial L}{\partial \mathbf{c}_t} \approx \frac{\partial L}{\partial \mathbf{c}_{t-1}}
\quad\Rightarrow\quad
\text{gradiente flui sem decair}
\]



\end{frame}


\begin{frame}{Hochreiter \& Schmidhuber (1997) — LSTM (cont.) \cite{hochreiter1997}}


  \begin{center}
    \includegraphics[width=0.55\textwidth]{/home/ferna/cognite/apresentacao/figs/LSTM.png}
  \end{center}
\medskip
\scriptsize
\textbf{Resumo:}
\begin{itemize}
  \item A LSTM adiciona \textbf{portas} para controlar escrita, esquecimento e leitura da memória.
  \item O estado de célula $\mathbf{c}_t$ funciona como \textbf{rota de gradiente preservada}.
  \item Resolve o problema identificado por \textbf{Hochreiter (1991)}.
\end{itemize}
\end{frame}

\begin{frame}{Schuster \& Paliwal (1997) — RNN Bidirecional (BiRNN) \cite{schuster1997}}
\footnotesize

\textbf{Ideia central:} processar a sequência \textbf{nos dois sentidos} — 
um fluxo temporal \textbf{forward} ($t=1\to T$) e outro \textbf{backward} ($t=T\to 1$),
permitindo que cada posição tenha acesso ao \textbf{passado e ao futuro}.

\medskip
\textbf{Equações da BiRNN:}
\[
\overrightarrow{h}_t = \phi\!\left(W_x^{\rightarrow}\,x_t + W_h^{\rightarrow}\,\overrightarrow{h}_{t-1} + b^{\rightarrow}\right)
\]

\[
\overleftarrow{h}_t = \phi\!\left(W_x^{\leftarrow}\,x_t + W_h^{\leftarrow}\,\overleftarrow{h}_{t+1} + b^{\leftarrow}\right)
\]

\textbf{Fusão (saída no tempo $t$):}
\[
h_t = [\,\overrightarrow{h}_t \,;\, \overleftarrow{h}_t\,]
\quad\text{(concatenação, mais comum)}
\]

\medskip
\textbf{Interpretação:}
\begin{itemize}
  \item A rede \textbf{forward} captura contexto do \textbf{passado}.
  \item A rede \textbf{backward} captura contexto do \textbf{futuro}.
  \item A combinação fornece \textbf{representações mais ricas por tempo} $t$.
\end{itemize}

\medskip
\textbf{Aplicações típicas:}
\begin{itemize}
  \item Rotulação de sequência (POS tagging, NER, chunking)
  \item Reconhecimento de fala e alinhamento temporal (CTC, seq2seq)
  \item Modelos de atenção e embeddings contextuais (pré-transformer)
\end{itemize}

\end{frame}


\begin{frame}{Graves \& Schmidhuber (2005) - BiLSTM — Bidirectional LSTM \cite{graves2005}}
\footnotesize

\textbf{Ideia central:} aplicar uma \textbf{LSTM} no sentido \textbf{forward} e outra no sentido \textbf{backward},
de modo que cada passo temporal tenha acesso a \textbf{passado + futuro}.

\medskip
\textbf{LSTM forward:}
\[
\overrightarrow{h}_t,\,\overrightarrow{c}_t = \text{LSTM}_{\rightarrow}(x_t, \overrightarrow{h}_{t-1}, \overrightarrow{c}_{t-1})
\]

\textbf{LSTM backward:}
\[
\overleftarrow{h}_t,\,\overleftarrow{c}_t = \text{LSTM}_{\leftarrow}(x_t, \overleftarrow{h}_{t+1}, \overleftarrow{c}_{t+1})
\]

\textbf{Fusão (saída no tempo $t$):}
\[
h_t = [\,\overrightarrow{h}_t \,;\, \overleftarrow{h}_t\,]
\quad\text{(concatenação, mais comum)}
\]

\medskip
\textbf{Interpretação:}
\begin{itemize}
  \item \(\overrightarrow{h}_t\): informações do \textbf{passado}.
  \item \(\overleftarrow{h}_t\): informações do \textbf{futuro}.
  \item A fusão fornece uma representação \textbf{contextual bidirecional}.
\end{itemize}



\end{frame}

\begin{frame}{BiLSTM — Long Short-Term Memory Bidirecional (cont.) \cite{graves2005}}


  \scriptsize
 
      \centering
      \textbf{BiLSTM}   
      \includegraphics[width=0.55\linewidth]{/home/ferna/cognite/apresentacao/figs/BiLSTM.png}
   \centering
   
   \medskip
\textbf{Por que utilizar BiLSTM?}
\begin{itemize}
  \item Mantém a \textbf{memória de longo prazo} da LSTM.
  \item Excelente para tarefas onde o \textbf{contexto futuro é relevante}.
  \item Muito utilizada em \textbf{NLP} (POS tagging, NER, parsing), \textbf{fala}, \textbf{bio-sinais}.
\end{itemize}
\end{frame}

\begin{frame}{Cho et al. (2014) — GRU (Gated Recurrent Unit) \cite{cho2014}}
\footnotesize

\textbf{Ideia central:} simplificar a LSTM mantendo a capacidade de \textbf{memória de longo prazo},
reduzindo o número de portas e \textbf{sem estado de célula separado} ($h_t$ = memória).

\medskip
\textbf{Equações da GRU:}
\[
\mathbf{z}_t = \sigma(W_{xz}\mathbf{x}_t + W_{hz}\mathbf{h}_{t-1} + \mathbf{b}_z)
\quad\text{(porta de atualização)}
\]

\[
\mathbf{r}_t = \sigma(W_{xr}\mathbf{x}_t + W_{hr}\mathbf{h}_{t-1} + \mathbf{b}_r)
\quad\text{(porta de reset)}
\]

\[
\tilde{\mathbf{h}}_t = \tanh\!\left(W_{xh}\mathbf{x}_t + W_{hh}(\mathbf{r}_t \odot \mathbf{h}_{t-1}) + \mathbf{b}_h\right)
\quad\text{(conteúdo candidato)}
\]

\[
\mathbf{h}_t = (1-\mathbf{z}_t)\odot \mathbf{h}_{t-1} + \mathbf{z}_t \odot \tilde{\mathbf{h}}_t
\quad\text{(estado oculto atualizado)}
\]

\medskip
\textbf{Interpretação:}
\begin{itemize}
  \item $\mathbf{z}_t$ decide \textbf{quanto do passado manter}.
  \item $\mathbf{r}_t$ decide \textbf{quanto ignorar do passado} ao propor novo conteúdo.
  \item Não possui estado de célula $c_t$: \textbf{memória é mantida em $h_t$}.
\end{itemize}



\end{frame}


\begin{frame}{Cho et al. (2014) — GRU (Gated Recurrent Unit) (cont.) \cite{cho2014}}


  \begin{center}
    \includegraphics[width=0.65\textwidth]{/home/ferna/cognite/apresentacao/figs/GRU.png}
  \end{center}
\medskip
\textbf{Por que GRU ficou popular?}
\begin{itemize}
  \item Menos parâmetros que LSTM → \textbf{mais leve}.
  \item Desempenho comparável em muitas tarefas sequenciais.
  \item Treino mais estável que RNN simples (não sofre vanishing severo).
\end{itemize}

\end{frame}

\begin{frame}{Chung, J. (2014) - BiGRU — Gated Recurrent Unit Bidirecional \cite{chung2014}}
\footnotesize

\textbf{Ideia central:} aplicar uma \textbf{GRU} no sentido \textbf{forward} e outra no sentido \textbf{backward}, 
combinando seus estados ocultos para obter \textbf{contexto passado + futuro}.

\medskip
\textbf{GRU forward:}
\[
\overrightarrow{h}_t = \text{GRU}_{\rightarrow}(x_t, \overrightarrow{h}_{t-1})
\]

\textbf{GRU backward:}
\[
\overleftarrow{h}_t = \text{GRU}_{\leftarrow}(x_t, \overleftarrow{h}_{t+1})
\]

\textbf{Fusão (saída no tempo $t$):}
\[
h_t = [\,\overrightarrow{h}_t \,;\, \overleftarrow{h}_t\,]
\quad\text{(concatenação)}
\]

\medskip

\textbf{Interpretação:}
\begin{itemize}
  \item \(\overrightarrow{h}_t\) captura dependências \textbf{anteriores} na sequência.
  \item \(\overleftarrow{h}_t\) captura dependências \textbf{futuras}.
  \item A concatenação fornece uma \textbf{representação mais rica} em cada passo temporal.
\end{itemize}



\end{frame}

\begin{frame}{Chung, J. (2014) - BiGRU — Gated Recurrent Unit Bidirecional (cont.)\cite{chung2014}}


  \scriptsize
 
      \centering
      \textbf{BiGRU}
     

      \includegraphics[width=0.55\linewidth]{/home/ferna/cognite/apresentacao/figs/BiGRU.png}

   \centering
   \medskip
\textbf{Por que utilizar BiGRU?}
\begin{itemize}
  \item Mantém as vantagens da GRU (menos parâmetros que LSTM).
  \item Boa estabilidade em sequências longas.
  \item Muito utilizada em \textbf{NLP, fala, anotação de sequência, NER, segmentação}.
\end{itemize}


\end{frame}


% ======================
\section{Arquiteturas Bidirecionais e Fusão Pós-BiRNN}
% ======================
\begin{frame}{Arquiteturas Bidirecionais: Acoplada vs. Desacoplada (``Zig–Zag'') \cite{serdyuk2018}}
\footnotesize

\textbf{Objetivo:} Capturar dependências temporais \textbf{passado $\leftrightarrow$ futuro}.

\vspace{0.3cm}
\textbf{1) BiRNN Acoplada (Convencional)}
\begin{itemize}
  \item \textbf{Forward} e \textbf{Backward} recebem \textbf{a mesma entrada} $x_t$.
  \item São processadas \textbf{em paralelo} e \textbf{independentes} entre si.
  \item A fusão ocorre \textbf{após} o cálculo dos dois estados:
\[
\overrightarrow{h}_t = f(x_t, \overrightarrow{h}_{t-1}),
\quad
\overleftarrow{h}_t = f(x_t, \overleftarrow{h}_{t+1}),
\quad
z_t = [\,\overrightarrow{h}_t ; \overleftarrow{h}_t\,].
\]
  \item \textbf{Usada apenas offline}, pois depende de informação futura.
\end{itemize}

\vspace{0.35cm}
\textbf{2) BiRNN Desacoplada (Zig--Zag / Teacher--Student)}
\begin{itemize}
  \item O ramo \textbf{backward é treinado primeiro} (usa contexto futuro).
  \item Depois, a rede \textbf{forward é treinada usando} $x_t$ \textbf{e} a representação do backward como \textbf{sinal professor}:
\[
\overleftarrow{h}_t = f(x_t, \overleftarrow{h}_{t+1}) \quad \text{(treino offline)}
\]
\[
\overrightarrow{h}_t = g(x_t, \overrightarrow{h}_{t-1},\, \overleftarrow{h}_t) \quad \text{(treino do aluno)}
\]
  \item Na \textbf{inferência}, somente $\overrightarrow{h}_t$ é usado → \textbf{modelo causal}.
\end{itemize}

\vspace{0.35cm}
\centering
\textit{\small Acoplada = duas direções independentes em paralelo. \\
Desacoplada = backward como \textbf{professor} → forward \textbf{aprende a prever sem ver o futuro}.}
\end{frame}

\begin{frame}{Técnicas de Fusão Bidirecional pós-BiRNN:}

  \begin{itemize}
    \item Concatenação Direta
    \item Fusão gated
    \item Fusão recorrente (Fuser RNN)
  \end{itemize}
\end{frame}

\begin{frame}{Fusão Bidirecional pós-BiRNN: Concatenação direta \cite{schuster1997,graves2005}}
  Após processar a sequência com uma BiRNN (BiLSTM/BiGRU), precisamos combinar as
  representações das duas direções em um vetor único para cada passo de tempo $t$
  \[
    z_t = [\,h_t^{\rightarrow} ; h_t^{\leftarrow}\,]
  \]
  \begin{itemize}
    \item Simples, padrão em BiLSTM/BiGRU.
    \item Mantém toda a informação das duas direções.
    \item Limitação: trata passado e futuro como igualmente relevantes, sem controle dinâmico.
  \end{itemize}
\end{frame}

\begin{frame}{Fusão Bidirecional pós-BiRNN: Fusão \textit{gated} (seleção dinâmica) \cite{serdyuk2018}}
  \vspace{0.3cm}
  \[
    g_t = \sigma\big(W_g [\,h_t^{\rightarrow} ; h_t^{\leftarrow}\,] + b_g\big)
  \]
  \[
    z_t = g_t \odot h_t^{\rightarrow} \;+\; (1-g_t) \odot h_t^{\leftarrow}
  \]
  \scriptsize
  \[
    \begin{aligned}
      &g_t &: \text{vetor de pesos entre 0 e 1 (gate dinâmico)} \\
      &W_g, b_g &: \text{parâmetros aprendidos} \\
      &\odot &: \text{produto elemento a elemento}
    \end{aligned}
  \]

  \normalsize
  \begin{itemize}
    \item A rede aprende quando confiar mais no histórico causal ($h_t^{\rightarrow}$)
          ou no contexto futuro ($h_t^{\leftarrow}$).
     \begin{itemize}
\item $g_t \!\approx\! 1$ → usa mais $h_t^{\rightarrow}$ (passado);
\item $g_t \!\approx\! 0$ → usa mais $h_t^{\leftarrow}$ (futuro);
\item $0<g_t<1$ → mistura adaptativa.
\end{itemize}
          \item Melhora robustez em sinais ruidosos.
    \item A fusão gated aprende automaticamente quanto de passado ou futuro é relevante em cada passo.
  \end{itemize}
\end{frame}

\begin{frame}{Fusão Bidirecional pós-BiRNN: Fusão recorrente (\textit{fuser RNN}) \cite{serdyuk2018}}
    \vspace{0.3cm}

  \[
    z_t = [\,h_t^{\rightarrow} ; h_t^{\leftarrow}\,]
  \]
  \[
    u_t = \text{GRU}_\text{fuser}(z_t, u_{t-1})
  \]
  \begin{itemize}
  \item Após concatenar as direções da BiRNN, uma GRU (ou LSTM) unidirecional processa a sequência fundida $z_t$.
     \item Gera uma representação \textbf{causal} $u_t$, pois depende apenas de $u_{t-1}$ e $z_t$.
  \item Atua como um \textbf{refinador temporal}: suaviza ruídos da BiRNN, estabiliza a dinâmica e acumula contexto adicional ao longo do tempo.
   \item Permite treinar com \textbf{contexto bidirecional} (offline), mas \textbf{implantar apenas um modelo causal} (online), ideal para aplicações em tempo real.
\end{itemize}


\end{frame}

\begin{frame}{Comparativo de Estratégias de Fusão Bidirecional}
  \scriptsize
  \renewcommand{\arraystretch}{1.3}
  \setlength{\tabcolsep}{3pt}

  \resizebox{\textwidth}{!}{
  \begin{tabular}{|p{2.2cm}|p{3.5cm}|p{4cm}|p{3.5cm}|}
    \hline
    \textbf{Método} &
    \textbf{Mecânica de Fusão} &
    \textbf{Vantagens Principais} &
    \textbf{Aplicações Típicas} \\ \hline

    \textbf{Concatenação} &
    Combina diretamente os estados das direções:
    \[
      z_t = [\,h_t^{\rightarrow} ; h_t^{\leftarrow}\,]
    \]
    &
    \begin{itemize}
      \item Simples e eficiente.
      \item Mantém toda a informação temporal.
      \item Facilmente integrável a MLPs ou CNNs.
    \end{itemize}
    &
    \begin{itemize}
      \item Classificação e regressão offline.
      \item Representação de contexto completo.
      \item Modelos base ou ablação.
    \end{itemize}
    \\ \hline

    \textbf{Fusão Gated} &
    Calcula um gate dinâmico:
    \[
      z_t = g_t \odot h_t^{\rightarrow} + (1-g_t) \odot h_t^{\leftarrow}
    \]
    onde \( g_t = \sigma(W_g [h_t^{\rightarrow};h_t^{\leftarrow}]) \)
    &
    \begin{itemize}
      \item Adapta a contribuição de cada direção por timestep.
      \item Robusto a ruído e desbalanceamento temporal.
      \item Oferece interpretabilidade (importância causal/anticausal).
    \end{itemize}
    &
    \begin{itemize}
      \item Séries ruidosas (sensores, texto clínico).
      \item Diagnóstico temporal e detecção de eventos.
    \end{itemize}
    \\ \hline

    \textbf{Fuser RNN} &
    Empilha uma RNN extra sobre a fusão:
    \[
      u_t = \text{GRU}_\text{fuser}([h_t^{\rightarrow};h_t^{\leftarrow}], u_{t-1})
    \]
    &
    \begin{itemize}
      \item Agrega contexto de longo prazo.
      \item Suaviza e refina a sequência.
      \item Permite distilação causal para deploy online.
    \end{itemize}
    &
    \begin{itemize}
      \item Modelos industriais/tempo real.
      \item Reconstrução e previsão contínua.
      \item “Teacher–student” bidirecional $\rightarrow$ causal.
    \end{itemize}
    \\ \hline
  \end{tabular}
  } % fecha resizebox

  \vspace{0.2cm}
  \centering
  \textit{\small Concat = simples, Gate = adaptativo, Fuser = refinamento temporal e causal.}
\end{frame}



% ======================
\section{Regularização e Estabilização em RNNs}
% ======================

% ======================
\begin{frame}{Regularização e Estabilização em RNNs}
\footnotesize

O treinamento de \textbf{Redes Neurais Recorrentes (RNNs)} exige técnicas específicas
para lidar com \textbf{instabilidades de gradiente} e \textbf{overfitting},
mantendo a coerência temporal e a convergência estável do modelo.

\medskip
\textbf{Objetivos principais:}
\begin{itemize}
  \item Evitar explosão/vanishing de gradientes.
  \item Melhorar generalização em sequências longas.
  \item Estabilizar ativações internas e acelerar o aprendizado.
\end{itemize}

\medskip
\textbf{Técnicas essenciais:}
\begin{enumerate}
  \item \textbf{Gradient Clipping \cite{pascanu2013}} — limita a norma do gradiente (\( \|g\|_2 \leq c \)) para evitar atualizações instáveis.  
        
  \item \textbf{Variational Dropout \cite{gal2016}} — aplica a mesma máscara de dropout ao longo do tempo, preservando a coerência temporal.  
        
  \item \textbf{Layer Normalization \cite{ba2016}} — normaliza as ativações por camada e timestep, reduzindo flutuações internas e melhorando a estabilidade numérica.  
       
\end{enumerate}

\medskip
Essas estratégias combinadas formam a base para o treinamento robusto de RNNs,
LSTMs e GRUs em contextos de sequências longas e ruído nos dados.

\end{frame}
% ======================

\begin{frame}{Pascanu, Mikolov \& Bengio (2013) — Gradient Clipping \cite{pascanu2013}}
\scriptsize

\textbf{Motivação:}  
Em redes recorrentes (RNNs) e arquiteturas profundas, o problema do \textbf{exploding gradient}
faz com que os gradientes cresçam descontroladamente, causando instabilidade numérica e divergência no treinamento.

\medskip
\textbf{Ideia principal:}  
Limitar a norma do gradiente (\textit{gradient norm clipping}) mantendo o vetor dentro de um raio máximo \(c\).

\vspace{0.2cm}

\textbf{1) Gradiente completo}
\[
g = \nabla_{\theta} L
\]
\begin{itemize}
  \item \(g\): vetor de derivadas da perda em relação aos parâmetros.
  \item Norma L2: \(\|g\|_2 = \sqrt{\sum_i (\partial L/\partial \theta_i)^2}\).
\end{itemize}

\vspace{0.2cm}

\textbf{2) Regra de clipping}
\[
\text{se } \|g\|_2 > c,\qquad g \leftarrow \frac{c}{\|g\|_2}\, g
\]
\begin{itemize}
  \item Fator de reescala: \(\alpha = \frac{c}{\|g\|_2} < 1\).
  \item Preserva a \textbf{direção} do gradiente; reduz apenas sua \textbf{magnitude}.
\end{itemize}

\vspace{0.2cm}

\textbf{3) Resultado do clipping}
\[
\left\|\frac{c}{\|g\|_2} g \right\|_2 = c
\]
\begin{itemize}
 \item Previne saltos muito grandes na superfície de perda.  
 \item Garante que a atualização permaneça num “raio seguro” em torno dos parâmetros.  
 \item Contribui para estabilidade numérica, e facilita o treinamento de redes profundas ou recorrentes.
\end{itemize}
\end{frame}



\begin{frame}{Gal \& Ghahramani (2016) - Variational Dropout \cite{gal2016}}
\footnotesize

\textbf{Problema do dropout tradicional em RNNs:}\\
Se uma \textbf{máscara diferente} for aplicada a cada timestep:
  
\begin{itemize}
  \item o ruído muda a cada timestep,
  \item \textbf{desestabiliza a memória} da rede, pois não há aplicação nas conexões recorrentes.
\end{itemize}

\medskip

\textbf{Solução: Variational Dropout}
\begin{itemize}
  \item A \textbf{mesma máscara de dropout} é mantida para todos os timesteps, inclusive nas conexões recorrentes.
  \item Preserva a coerência temporal, mantendo a mesma estrutura ativa ou inativa durante toda a sequência, estabilizando o aprendizado.
  \item Reduz overfitting sem destruir o estado recorrente.
\end{itemize}

\centering
\includegraphics[width=0.80\linewidth]{/home/ferna/CPE727-2025-03/Seminarios/6 - RNN/apresentacao/figs/variationalRNN.png}
% (se você não tiver a imagem, eu gero uma versão em estilo hand-drawn para slide)


\end{frame}


\begin{frame}{Gal \& Ghahramani (2016) - Variational Dropout (cont.) \cite{gal2016}}

  Em frameworks modernos:
  \[
    h_t = f\big(W_{hh}\,(d_h \odot h_{t-1}) + W_{xh}\,x_t\big)
  \]

  \scriptsize
  \[
  \begin{aligned}
    &h_t &: \text{estado oculto no tempo } t \\
    &x_t &: \text{entrada no tempo } t \\
    &W_{hh}, W_{xh} &: \text{matrizes de pesos recorrentes e de entrada} \\
    &d_h &: \text{máscara de dropout, } d_h \sim \text{Bernoulli}(p) \\
    &p &: \text{probabilidade de retenção (ex: } p = 0.8\text{)} \\
    &\odot &: \text{produto elemento a elemento} \\
    &f(\cdot) &: 
       \begin{cases}
         \tanh(\cdot), & \text{em uma RNN simples} \\
         \text{gates + }\tanh(\cdot), & \text{em uma GRU} \\
         \text{combinação das portas } (i,f,o,g), & \text{em um LSTM}
       \end{cases}
  \end{aligned}
  \]

\end{frame}


\begin{frame}{Ba, Kiros \& Hinton (2016) - Layer Normalization (LN) \cite{ba2016}}
 \small
  \begin{itemize}
    \item Normaliza as ativações dentro de cada camada, em cada passo de tempo, de forma independente do tamanho do batch:
      \[
        \text{LN}(h_t) = \gamma \frac{h_t - \mu_t}{\sqrt{\sigma_t^2 + \varepsilon}} + \beta
\quad\text{com }\varepsilon>0 \text{ pequeno.}
      \]
      \scriptsize
      \[
      \begin{aligned}
        &h_t &: \text{vetor de ativações da camada no tempo } t \\
        &\mu_t, \sigma_t &: \text{média e desvio padrão das ativações em } h_t \\
        &\gamma, \beta &: \text{parâmetros aprendidos de escala e deslocamento}
      \end{aligned}
      \]
       \small
    \item Reduz instabilidade do gradiente e acelera a convergência, especialmente em sequências longas.
    \item Em LSTMs/GRUs, é aplicada nas transformações lineares de cada porta:
      \[
        i_t = \sigma(\text{LN}(W_i[h_{t-1},x_t]) + b_i)
      \]
      \scriptsize
      \[
      \begin{aligned}
        &i_t &: \text{porta de entrada (input gate)} \\
        &W_i &: \text{pesos da porta de entrada} \\
        &[h_{t-1},x_t] &: \text{concatenação do estado anterior e da entrada atual} \\
        &b_i &: \text{termo de viés (bias)}
      \end{aligned}
      \]
      \small
      O mesmo processo é aplicado às demais portas (f, o, g), mantendo escalas consistentes entre todas as transformações internas.
  \end{itemize}
\end{frame}


\begin{frame}{Evolução para Arquiteturas Modernas}

\small
\begin{itemize}
\item \textbf{2014 - Sequence-to-Sequence (Seq2seq) \cite{sutskever2014}:} 
Sutskever et al. introduzem arquitetura encoder-decoder com RNNs/LSTMs para tradução automática neural, revolucionando o campo
\item \textbf{2015 - Mecanismo de Atenção \cite{bahdanau2015}:}
Bahdanau et al. propõem atenção para resolver limitação do vetor de contexto fixo, permitindo que o decoder "foque" em partes relevantes da entrada.
\item \textbf{2017 - Transformer \cite{vaswani2017}:}
Vaswani et al. introduzem arquitetura baseada puramente em atenção (self-attention), eliminando recorrência e permitindo paralelização massiva.
\item \textbf{2018 - BERT e GPT \cite{devlin2019_bert,radford2019_gpt2}:}
Modelos de linguagem pré-treinados baseados em Transformer (BERT para encoding bidirecional, GPT para geração autoregressiva) alcançam state-of-the-art em NLP.
\item \textbf{2020 - Large Language Models (LLMs) \cite{brown2020_gpt3,bommasani2021_foundation}:}
GPT-3, GPT-4, LLaMA, PaLM e outros LLMs com bilhões de parâmetros dominam tarefas de linguagem natural através de scaling e prompting.

\end{itemize}

\end{frame}



\begin{frame}{Comparação entre LSTM, GRU, BiLSTM e BiGRU}
  RNNs permanecem relevantes em nichos específicos: dispositivos embarcados com recursos limitados, processamento em tempo real com baixa latência, e aplicações onde paralelização não é crítica. \\ 
  
  \scriptsize
  \begin{table}[h]
    \centering
    \resizebox{\textwidth}{!}{%
      \begin{tabular}{@{}lcccc@{}}
        \toprule
        \textbf{Característica} & \textbf{LSTM} & \textbf{GRU} & \textbf{BiLSTM} & \textbf{BiGRU} \\ 
        \midrule
        Portas & 3 (forget, input, output) & 2 (reset, update) & 3 (duas direções) & 2 (duas direções) \\
       Estados internos & $C_t$, $h_t$ & $h_t$ & $C_t$, $\overrightarrow{h}_t$, $\overleftarrow{h}_t$ & $\overrightarrow{h}_t$, $\overleftarrow{h}_t$ \\
        Contexto capturado & Passado & Passado & Passado e futuro & Passado e futuro \\
        Nº de parâmetros & Alto & $\sim$25\% menor & $\sim$2× o da LSTM & $\sim$2× o da GRU \\
        Velocidade de treino & Mais lenta & Mais rápida & Mais lenta (duas passagens) & Razoável (duas passagens) \\
        Capacidade de modelagem & Alta & Boa & Muito alta & Alta \\
        Aplicações típicas & Séries longas & Modelos leves & NLP, voz, texto bidir. & NLP leve, embeddings \\
        \bottomrule
      \end{tabular}%
    }
  \end{table}

  \vspace{0.3cm}
  \small
  \textbf{Resumo:}  
  Modelos bidirecionais (BiLSTM/BiGRU) exploram dependências temporais em ambas as direções.  
  GRU e BiGRU são mais leves; LSTM e BiLSTM tendem a capturar relações mais complexas.
  \end{frame}


\begin{frame}{Conclusões Principais}
\footnotesize

\textbf{1) Contribuição Conceitual das RNNs}
\begin{itemize}
  \item Introduzem a ideia de \textbf{estado oculto} como \textbf{memória dinâmica}.
  \item Permitem modelar séries temporais, fala, texto e qualquer dado sequencial.
  \item Fundamentam a passagem de redes estáticas → modelos temporais.
\end{itemize}

\medskip
\textbf{2) Contribuição Matemática}
\begin{itemize}
  \item BPTT torna explícita a \textbf{dependência temporal no gradiente}.
  \item Vanishing/Exploding não é falha de implementação — é da \textbf{recorrência}.
  \item LSTM e GRU criam \textbf{caminhos estáveis} para o gradiente ao longo do tempo.
\end{itemize}

\medskip
\textbf{3) Arquiteturas Modernas}
\begin{itemize}
  \item BiLSTM/BiGRU incorporam \textbf{contexto futuro} → melhores embeddings.
  \item Ainda são eficazes quando há \textbf{causalidade}, \textbf{baixa latência} ou \textbf{poucos dados}.
  \item \textbf{Transformers não substituem RNNs}: estendem a ideia com \textbf{atenção global}.
\end{itemize}

\medskip
\textbf{Mensagem Final:}
RNNs são o \textbf{ponto de transição} entre redes neurais clássicas e modelos de atenção.
Entender suas equações é entender \textbf{o alicerce} do processamento moderno de sequências.
\end{frame}

% ======================
\section{Casos de Uso em Séries Temporais e Processamento de Linguagem Natural}
% ======================

\begin{frame}{Caso de Uso em Séries Temporais}
\footnotesize

\textbf{Problema:}  
Modelar séries temporais industriais (compressores) com amostragem irregular, dados faltantes e alta dimensionalidade.
Dois cenários: \textbf{predição} e \textbf{reconstrução} utilizando arquiteturas recorrentes (RNNs e BiRNNs).

\medskip
\textbf{Estratégias adotadas:}
\begin{itemize}
  \item Avaliação de modelos clássicos e bidirecionais.
  \item Teste de três técnicas de fusão pós-BiRNN (concat/ gate / GRU fuser).
  \item Uso de regularização temporal: layer normalization.
\end{itemize}

\medskip
\textbf{Base de dados:}  
Cognite — Sinais reais de sensores de um compressor industrial offshore.

\vspace{0.25cm}
\centering
\includegraphics[width=0.45\linewidth]{/home/ferna/CPE727-2025-03/Seminarios/6 - RNN/apresentacao/figs/12series.png}

\vspace{0.2cm}
\scriptsize
\textit{Exemplo de séries reais utilizadas no estudo (12 sinais simultâneos).}

\end{frame}


\begin{frame}{Caso de Uso em Séries Temporais - Metodologia de Treinamento e Avaliação}
  \textbf{Figura de Mérito:}
  \begin{itemize}
    \item Micro MSE (Mean Squared Error)
  \end{itemize}

  \textbf{Setup de Treino:}
  \begin{itemize}
    \item Divisão de dados: 60\% treino, 20\% validação, 20\% teste.
    \item Otimizador: Adam W 
    \item Warm up
    \item Schedular
    \item Weight decay (L2 regularization).
    \item Early stopping com paciência de 20 épocas para Predição e 50 épocas para Reconstrução.
  \end{itemize}

\end{frame}

\begin{frame}{Caso de Uso em Séries Temporais - Metodologia de Treinamento e Avaliação}

  \begin{itemize}
    % ====================== PREDIÇÃO ======================
    \item \textbf{Predição:}
      \begin{itemize}
        \item Loss: MSE no último timestamp de cada janela
        \[
          \mathcal{L}
          =
          \frac{1}{\sum_{t=1}^{T} m_t}
          \sum_{t=1}^{T}
            m_t \cdot (x_t - \hat{x}_t)^2
        \]
      \end{itemize}

    % ==================== RECONSTRUÇÃO ====================
    \item \textbf{Reconstrução:}
      \begin{itemize}
        \item Inserção de difusão de ausência (missingness diffusion)
        \item Loss: último timestamp da janela \\
              L1 — NLL Gaussiano ponderado por \(\hat\lambda_t\):
        \[
          \text{sse}_t \;=\; \sum_{c}\big(\hat x_{t,c}-x_{t,c}\big)^2, 
          \qquad
          \hat\lambda_t \;=\; \mathrm{softplus}\!\big(W_\lambda h_t + b_\lambda\big)
        \]
        \[
          \log p(x_t\mid h_t)\;=\;-\tfrac12\,\hat\lambda_t\,\text{sse}_t 
          \;+\; \tfrac12\,n\,\log\hat\lambda_t 
          \;-\; \tfrac12\,n\log(2\pi)
        \]
        \[
          \mathcal{L}_1 \;=\; -\sum_{t}\log p(x_t\mid h_t)
        \]

        \item Layer Normalization
         
      \end{itemize}
  \end{itemize}

\end{frame}




\begin{frame}{Caso de Uso em Séries Temporais - Resultados MSE micro na Predição e Reconstrução}
\centering
 \includegraphics[width=\linewidth]{/home/ferna/CPE727-2025-03/Seminarios/6 - RNN/apresentacao/figs/comparcao.png}
     
 A reconstrução alcança resultados melhores que a predição, apesar de ser mais complexa, pois além de informar o que a predição já faz,
 a reconstrução recupera os valores ausentes da série temporal por difusão.

\end{frame}

\begin{frame}{Resultados - MSE micro na Predição e Reconstrução}
\small
\begin{columns}[T,onlytextwidth]

% -------------------------------- LEFT TABLE: Prediction --------------------------------
\begin{column}{0.48\textwidth}
\centering
\textbf{Predição}\\[-0.3mm]
\resizebox{\linewidth}{!}{%
\begin{tabular}{lcc}
\toprule
\textbf{Configuração} & \textbf{Média} & \textbf{Desvio padrão} \\
\midrule
\textbf{BiGRU }           & 0.560974 & {$\pm\,0.350897$} \\
BiGRU coupled                      & 0.568346 & {$\pm\,0.348221$} \\
GRU                                & 0.592064 & {$\pm\,0.329763$} \\
BiGRU coupled gate                 & 0.605323 & {$\pm\,0.385184$} \\
BiLSTM coupled                     & 0.661299 & {$\pm\,0.370403$} \\
BiLSTM                             & 0.663868 & {$\pm\,0.426597$} \\
BiLSTM coupled gate                & 0.669857 & {$\pm\,0.359655$} \\
LSTM                               & 0.679496 & {$\pm\,0.365859$} \\
BiGRU coupled GRU Fuser            & 0.686827 & {$\pm\,0.320792$} \\
BiLSTM coupled GRU Fuser           & 0.741522 & {$\pm\,0.380521$} \\

\bottomrule
\end{tabular}
}% <-- fecha o \resizebox
\end{column}

% -------------------------------- RIGHT TABLE: Reconstruction --------------------------------
\begin{column}{0.48\textwidth}
\centering
\textbf{Reconstrução}\\[-0.3mm]
\resizebox{\linewidth}{!}{%
\begin{tabular}{lcc}
\toprule
\textbf{Configuração} & \textbf{Média} & \textbf{Desvio padrão} \\
\midrule
\textbf{BiGRU GRU Fuser}           & 0.123370 & {$\pm\,0.040885$} \\
BiLSTM gate                        & 0.133257 & {$\pm\,0.035817$} \\
BiGRU                              & 0.133886 & {$\pm\,0.035452$} \\
BiLSTM                             & 0.137234 & {$\pm\,0.040035$} \\
BiLSTM coupled GRU Fuser           & 0.138969 & {$\pm\,0.043256$} \\
BiGRU coupled gate                 & 0.140517 & {$\pm\,0.037876$} \\
BiGRU coupled                    & 0.141365 & {$\pm\,0.038309$} \\
BiLSTM coupled gate                & 0.143339 & {$\pm\,0.042448$} \\
BiLSTM coupled                      & 0.144023 & {$\pm\,0.033444$} \\
BiGRU coupled GRU Fuser            & 0.146325 & {$\pm\,0.038838$} \\
BiLSTM GRU Fuser                   & 0.153163 & {$\pm\,0.038024$} \\
BiGRU gate                         & 0.161285 & {$\pm\,0.027645$} \\
GRU                                & 0.291358 & {$\pm\,0.043240$} \\
LSTM                               & 0.392539 & {$\pm\,0.044289$} \\
\bottomrule
\end{tabular}
}% <-- fecha o \resizebox
\end{column}

\end{columns}

\vspace{1.5mm}
{\scriptsize
\textcolor{blue!60}{\(\star\) Regra dos Dois Desvios Padrão:  
valores dentro de \(\mu \pm 2\sigma\) correspondem aproximadamente ao intervalo de confiança de 95\%.}
}
\end{frame}


% ================= SLIDE 5 =================
\begin{frame}{Cao de Uso em Processamento de Linguagem Natural: Hierarchical Attention-based Recurrent Neural Network - HARNN \cite{huang2019_hierarchical_multilabel_text_classification}}
  \small
  \textbf{Problema:} Classificação multirrótulo hierárquica na DBpedia (L1=9, L2=70, L3=219). \\
    \textbf{Desafio:} prever vários rótulos mantendo coerência pai→filho e lidando com desbalanceamento.
  \centering
  \includegraphics[width=0.9\linewidth]{/home/ferna/CPE727-2025-03/Seminarios/6 - RNN/apresentacao/figs/harnn.png}
\end{frame}


% ================= SLIDE 4 =================
\begin{frame}{Cao de Uso em Processamento de Linguagem Natural: Hierarchical Attention-based Recurrent Neural Network - HARNN \cite{huang2019_hierarchical_multilabel_text_classification}}
  \small

  \begin{itemize}
    \item \textbf{1) Documentation Representing Layer (DRL)}
      \begin{itemize}
        \item Cada palavra vira um vetor de embedding e a BiLSTM lê a frase nos dois sentidos,
              produzindo uma representação contextual do documento.
        \item Cada classe da árvore hierárquica tem seu embedding de rótulo, que funciona como
              uma \emph{consulta de atenção} ligando o documento à hierarquia.
      \end{itemize}

    \vspace{0.25cm}
    \item \textbf{2) Hierarchical Attention-based Recurrent Layer (HAM)}
      \begin{itemize}
        \item \textbf{TCA}: destaca as palavras mais relevantes para cada categoria do nível,
              comparando as palavras do texto com o vetor (embedding) da própria categoria
              e montando um ``resumo por categoria''.
        \item \textbf{CPM}: usa esses resumos para prever as categorias do nível.
        \item \textbf{CDM}: passa sinais do nível de cima para o de baixo, ajudando a manter
              coerência (L1 guia L2; L2 guia L3).
      \end{itemize}

    \vspace{0.25cm}
    \item \textbf{3) Hybrid Predicting Layer (HPL)}
      \begin{itemize}
        \item Junta a visão local (por nível) com a visão global do documento.
        \item Gera a decisão final, equilibrando consistência hierárquica e detalhe específico.
      \end{itemize}
  \end{itemize}

\end{frame}



% ================= SLIDE 17 =================
\begin{frame}{Resultados -- Dados de teste}
  \scriptsize
  \begin{columns}[T,onlytextwidth]

    % ----- Coluna esquerda: tabelas -----
    \begin{column}{0.55\textwidth}
      \centering
      \textbf{Métricas micro}\\[0.2cm]
      \begin{tabular}{lccc}
        \toprule
        \textbf{Método} & F1$_{\text{micro}}$ & Prec$_{\text{micro}}$ & Rec$_{\text{micro}}$ \\
        \midrule
        REG LOG          & 0.8123 & 0.8123 & 0.8123 \\
        XGBoost          & 0.9182 & 0.9312 & 0.9060 \\
        HARNN adaptado & 0.9701 & \textbf{0.9764} & 0.9639 \\
        HARNN campeã     & \textbf{0.9739} & 0.9755 & \textbf{0.9722} \\
        \bottomrule
      \end{tabular}

      \vspace{0.35cm}
      \textbf{Métricas macro}\\[0.2cm]
      \begin{tabular}{lccc}
        \toprule
        \textbf{Método} & F1$_{\text{macro}}$ & Prec$_{\text{macro}}$ & Rec$_{\text{macro}}$ \\
        \midrule
        REG LOG          & 0.7114 & 0.7335 & 0.7314 \\
        XGBoost          & 0.8640 & 0.9055 & 0.8322 \\
        HARNN adaptado & 0.9475 & \textbf{0.9617} & 0.9373 \\
        HARNN campeã     & \textbf{0.9548} & 0.9586 & \textbf{0.9524} \\
        \bottomrule
      \end{tabular}

      \vspace{0.35cm}
      \textbf{Subset accuracy}\\[0.2cm]
      \begin{tabular}{lc}
        \toprule
        \textbf{Método} & Subset\_Acc \\
        \midrule
        REG LOG          & 0.7355 \\
        XGBoost          & 0.7853 \\
        HARNN adaptado & 0.9266 \\
        HARNN campeã     & \textbf{0.9412} \\
        \bottomrule
      \end{tabular}
    \end{column}

    % ----- Coluna direita: interpretação -----
    \begin{column}{0.45\textwidth}
      \begin{itemize}
        \item \textbf{Ganho claro no nível micro}:\\
              HARNN campeã supera todas as demais em F1 e Recall.
        \item \textbf{Macro com melhora robusta}:\\
              $+0{,}0908$ em F1 vs.\ XGBoost; $+0{,}2434$ vs.\ REG LOG e
              vs.\ adaptado $+0{,}0073$ --- melhor equilíbrio entre classes.
        \item \textbf{Subset accuracy bem superior}:\\
              0{,}9412 ($+15{,}6$ pp sobre XGBoost; $+20{,}6$ pp sobre REG LOG;
              $+1{,}46$ pp sobre HARNN adaptado).
        \item \textbf{Trade-off precisão/recall}:\\
              HARNN campeã tem precisão levemente menor
              ($-0{,}09$ pp micro; $-0{,}31$ pp macro) em relação ao adaptado,
              mas o recall maior ($+0{,}83$ pp micro; $+1{,}51$ pp macro)
              melhora cobertura de classes de cauda e aumenta F1.
      \end{itemize}
    \end{column}

  \end{columns}
\end{frame}


\begin{frame}[allowframebreaks]{Referências}
  \tiny
  \bibliographystyle{ieeetr}   % estilo numérico
  \bibliography{refs}          % usa o arquivo refs.bib
\end{frame}

\end{document}
